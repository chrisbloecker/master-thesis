\chapter{Related Work}
We will give an overview on related work that has been done in this field. First, we briefly introduce two process calculi, i.e. formalisms that allow to describe the behaviour of processes. After that, we will discuss how these formal concepts can be used in programming.

\section{Process calculi}
Process calculi\index{Process!calculus} are formalisms that can be used to describe processes and their behaviour in a concurrent system on a high level. The description is done in a mathematical way, similar to algebraic structures\index{Algebraic!structure}, and independently from an implementation \cite{}. Usually process calculi operate on abstract processes and provide a set of operators and combinators which can be used to compose processes. By introducing this kind of formalism, process calculi make it possible to reason about processes and perform (equivalence) transformations on them, e.g. to obtain a more optimised\footnote{Optimised according to a defined criterion.} representation \cite{}. In \cite{Hoare:2012:LPU:2368298.2368301}, an algebraic model, which can be used to derive different process calculi, has been introduced. The common, generalised model for various process calculi shows that these calculi are essentially equivalent and can be used to express the same thing.

\subsection{Calculus of Communicating Systems\index{Calculus of Communicating Systems}\index{CCS}}
The Calculus of Communicating Systems, short \textsc{CCS}, is a process algebra introduced by Robin Milner\index{Milner, Robin} in 1982 \cite{Milner:1982:CCS:539036}. \textsc{CCS} was one of the first process calculi in the history of computer science. It comprises a relatively small set of combinators\index{Combinator} which are used to describe process behaviour. We give an overview over most of them and illustrate how to use them. Note that our introduction to \textsc{CCS} is not complete and we only want to show the general idea.

In \textsc{CCS}, processes are composed of actions $\mathcal{A}$ and processes $\mathcal{P}$. In the following, let $a \in \mathcal{A}$ be actions and $P, Q, R \in \mathcal{P}$ be processes.

Processes in \textsc{CCS} are defined inductively. The empty process $\emptyset$ is the process that does nothing. Processes can be prefixed with an action to obtain a new process. $a.P$ describes a process that can execute action $a$ before it continues as process $P$. We can use the empty process $\emptyset$ together with the prefixing of actions \enquote{.} to build a process that consists of an arbitrary long but finite sequence of actions.

The composition of two processes can be made with the choice operator \enquote{+} and the parallel operator \enquote{|}. $P + Q$ describes a process that proceeds as either $P$ or $Q$. With $P \,|\, Q$ we can describe that processes $P$ and $Q$ exist and the same time and execute concurrently. Both for \enquote{+} and \enquote{|} the laws of associativity and commutativity hold as shown in the following equations:
\begin{eqnarray*}
  P + \left( Q + R \right) & = & \left( P + Q \right) + R \\
  P \,|\, \left( Q \,|\, R \right) & = & \left( P \,|\, Q \right) \,|\, R \\
  P + Q & = & Q + P \\
  P \,|\, Q & = & Q \,|\, P.
\end{eqnarray*}
Furthermore, \enquote{$|$} distributes over \enquote{$+$}:
\begin{eqnarray*}
  P \,|\, \left( Q + R \right) & = & \left( P \,|\, Q \right) + \left( P \,|\, R \right).
\end{eqnarray*}

\textsc{CCS} contains more than we have seen here, e.g. aliasing of processes and thereby introducing recursion. Sequential and parallel composition using the shown combinators as well as choice between processes is straight forward and intuitive.

\subsection{Communicating Sequential Processes\index{Communicating Sequential Processes}\index{CSP}}
Communicating Sequential Processes, or short \textsc{CSP}, was developed by C.A.R. Hoare\index{Hoare, C.A.R.} and published in 1985 \cite{Hoare:1985:CSP:3921}. It arose around the same time as \textsc{CCS} and belongs to the circle of the first process calculi\index{Process!calculus}. We introduce a subset of \textsc{CSP} and show how to build processes using \textsc{CSP}, however we don't go too deep into detail, neither do we claim to provide a complete description.

In \textsc{CSP}, processes are composed of two types of primitives: sequential processes $\mathcal{P}$ and events $\mathcal{E}$. A process $P \in \mathcal{P}$ can observe an event\index{Event} $e \in \mathcal{E}$ and react to it. It cannot take influence on events or manipulate them as they are indivisible. In the following, let $P, Q \in \mathcal{P}$ be processes and $a,b \in \mathcal{E}$ be events.

The prefix combinator \enquote{$\to$} takes an event $a$ and a process $P$. $\left( a \to P \right)$ creates a new process that waits until it observes $a$ and then behaves like $P$. A process that repeatedly waits to observe $a$ and then behaves like $P$ can be described using recursion: $P = \left( a \to P \right)$.

With the choice operator \enquote{$|$} we can construct a process that allows two flows of control. The process $\left( a \to P \,|\, b \to Q \right)$ waits until it either observes $a$ and then behaves like $P$ or it waits until it observes $b$ and then behaves like $Q$. Note that the prefix operator binds stronger than the choice operator and that the choice operator has commutative property.

The parallel operator \enquote{$||$} allows us to put processes together such that they run parallel. The process obtained by $\left( P \,||\, Q \right)$ waits for an event that both $P$ and $Q$ can observe and then behaves like $P$ and $Q$ concurrently\footnote{Note there is a difference between concurrency and nondeterminism.}. Let $P = \left( a \to P \,|\, b \to a \to P \right)$ and $Q = a \to Q$ then $\left( P \,||\, Q \right) = \left( a \to P \,||\, a \to Q \right)$, since $Q$ cannot observe $b$ and thus the second choice from $P$ has to be omitted.

In contrast to the parallel operator \enquote{$||$}, the interleaving operator \enquote{$|||$} can be used to put processes together to execute concurrently, independently from which events they are able to observe. The process $\left( P \,|||\, Q \right)$ behaves like $P$ and $Q$ at the same time. If an event occurs that either of the processes can observe, the according process does so. If an event occurs that both processes, $P$ and $Q$, are able to observe, then the choice for which of them observes it is made nondeterministically.

We have now seen some basic combinators of \textsc{CSP} and how to use them in order to create composed processes based on primitive processes and events. We can construct processes that run in parallel and make their execution dependant on observable events. The formalism of \textsc{CSP} allows reasoning about processes and transformations on them. \textsc{CSP} comes with many more operators and combinators than the ones we have introdcued, but as already mentioned we do not intend to give a complete description.

%Unfortunately in \textsc{CSP} there exists communication between processes that is implicitly represented as events. Since we want to eliminate explicit communication between processes, \textsc{CSP} is unsuitable for our purposes.

\section{Application in programming}
Current programming languages come with support for parallel and concurrent programming, either by employing a runtime system, a virtual machine or using a library. Syntactically, sequential and parallel composition as well as the choice between processes look differently in different languages, yet the underlying semantics are the same\footnote{Of course only the semantics concerning process composition. Not so the semantics regarding program evaluation.}. %This is why we don't dive into the concrete syntax of a particular programming language, but rather describe how to apply the general principle.

Two processes can be combined sequentially by running one of the processes first, waiting for its termination and then running the other one. Parallel composition can be done by running processes in different threads. A choice between processes can be made using the conventional choice operator (if-then-else) from sequential programming. Descriptions made in, e.g. \textsc{CCS} can be resembled using this approach.

However, in either case it is necessary to explicitly write the desired composition down. There is usually no direct way of composing processes and (re-)using them as first class values. This unfortunately results in lengthy, unmodular code. We would rather wish for a compact method, similar to the syntax of \textsc{CCS} and \textsc{CSP}, of performing process composition.

\subsection{Occam\index{Occam}}
The programming language \textsf{occam}\footnote{The reference manual for \textsf{occam} can be found at \url{http://www.wotug.org/occam/documentation/oc21refman.pdf}}, which first appeared in 1983, resembles the ideas of CSP closely. It incorporates direct implementations of the operations of choice between processes as well as parallel and sequential combination of processes. Furthermore it introduces named channels and both sending and receiving operations on them.

Sending data on a channel can be done using the sending operator \texttt{!}, receiving data from a channel can be done using the receiving operator \texttt{?}, see \lstref{lst:occam_send_receive}.

\begin{lstlisting}[language=Caml, caption=Sending data over a channel and receiving data from a channel in \textsf{occam}., label=lst:occam_send_receive, numbers=left, frame=bt]
channel ! out
channel ? in
\end{lstlisting}

In contrast to many other programming languages where sequential composition is implicit, \textsf{occam} has an operator for sequential composition of expressions: \texttt{SEQ}. It takes a list of processes and executes them sequentially as shown in \lstref{lst:occam_seq}.

\begin{lstlisting}[language=Caml, caption=Sequential composition of processes in \textsf{occam}., label=lst:occam_seq, numbers=left, frame=bt, firstnumber=3]
SEQ
  x := 3 * 7
  y := x * 2
\end{lstlisting}

Parallel composition can be done using the \texttt{PAR} combinator in \textsf{occam}. Like \texttt{SEQ}, it takes a list of processes and evaluates them concurrently as shown in \lstref{lst:occam_par}.

\begin{lstlisting}[language=Caml, caption=Parallel composition of processes in \textsf{occam}., label=lst:occam_par, numbers=left, frame=bt, firstnumber=6]
PAR
  x := f(a)
  y := g(b)
\end{lstlisting}

A non-deterministic\index{Non-determinism} choice between different alternatives of processes can be made using the \texttt{ALT} combinator. It takes a list of guarded processes and non-deterministically selects one of them for execution provided its guard signals readiness for the execution of the process. If none of the guards signals readiness, \texttt{ALT} waits until one of them does. A guard can be either a boolean expression, the action of reading from a channel or the combination of both.

\begin{lstlisting}[language=Caml, caption=Choice between process alternatives in \textsf{occam}., label=lst:occam_par, numbers=left, frame=bt, firstnumber=9]
ALT
  x > 3 & chan1 ? msg
    SEQ
      ...
  chan2 ? msg
    PAR
      ...
  ...
\end{lstlisting}