\chapter{Related Work}
We will give an overview on related work that has been done in this field. First, we briefly introduce two process calculi, i.e. formalisms that allow to describe the behaviour of processes. After that we will have a look at how these concepts are implemented in some current programming languages.

\section{Process calculi}
Process calculi are formalisms that can be used to describe processes and their behaviour in a concurrent system on a high level and independently from an implementation \cite{}. Usually process calculi operate on abstract processes and provide a set of operators and combinators which can be used to compose processes. This includes sending and receiving messages over communication channels, synchronisation with other processes and the sequence of actions to be performed \cite{}.

\subsection{CSP}

\subsection{$\pi$ calculus}

\section{Programming languages}

\subsection{Java}

\subsection{Erlang}

\subsection{Haskell}