\nomenclature[301]{$\choice{\_}{\_}{\_}$\dotfill}{Choice composition of processes}
\nomenclature[302]{$\parallel{\_}{\_}{\_}$\dotfill}{Parallel composition of processes}
\nomenclature[303]{$\sequence{\_}{\_}$\dotfill}{Sequential composition of processes}
\nomenclature[304]{$\repetition{\_}{\_}$\dotfill}{Repetition of a process}

\nomenclature[310]{\hfill}{}

\nomenclature[311]{$\mathcal{P}$\dotfill}{Set of all processes}
\nomenclature[312]{$B, C, P, Q, R, S, T$\dotfill}{Processes}
\nomenclature[313]{$\mathcal{T}$\dotfill}{Set of types}
\nomenclature[314]{$T_i$\dotfill}{Data type $i$}
\nomenclature[315]{$a,b,c,d,e$\dotfill}{Type variables}
\nomenclature[316]{$\rho$\dotfill}{Function that assigns a type signature to a process}
\nomenclature[317]{$\bot$\dotfill}{The undefined value}

\nomenclature[320]{\hfill}{}
\nomenclature[321]{$sem$\dotfill}{Function that gives the semantics of a process}
\nomenclature[322]{$f_P$\dotfill}{Intrinsic function of a basic process $P$}
\nomenclature[323]{$\circ$\dotfill}{Function composition}
\nomenclature[324]{$B\left(x\right)$\dotfill}{Shorthand notation for $\sem{B} \left( x \right) = True$}
\nomenclature[325]{$\overline{B\left(x\right)}$\dotfill}{Shorthand notation for $\sem{B} \left( x \right) = False$}
\nomenclature[326]{$Id$\dotfill}{The identity process}
\nomenclature[327]{$Err$\dotfill}{The error process}

\nomenclature[330]{\hfill}{}
\nomenclature[331]{$\equiv$\dotfill}{Process equivalence}
\nomenclature[332]{$P_{\left[ Q / R \right]}$\dotfill}{Substitution of sub-process $Q$ with $R$ in $P$}
\nomenclature[333]{$\overline{P}$\dotfill}{Process that is inverse to $P$}

\nomenclature[900]{\hfill}{}
\nomenclature[931]{iff\dotfill}{if and only if}


\chapter{The Calculus\index{Calculus}}
\label{chp:algebraic_model}
\label{chp:calculus}
In the previous chapter, we have seen two examples of process calculi\index{Process!Calculus} and how they work. In this chapter, we develop our own calculus and define its semantics\index{Semantics}. In contrast to \textsc{CCS} and \textsc{CSP}, our model does not involve non-determinism\index{Non-determinism} in the choice between processes. Furthermore, it does not involve any explicit notation of communication between processes.

First, let us define what we understand as a \textit{process}\index{Process}: a process is (a piece of) a computer program. It can run concurrently with other processes. A process receives an input\index{Input} and produces an output\index{Output}. Essentially, a process can be seen as a function\index{Function} and just like that, it is pure and free of side-effects\index{Side-effect}, it does not have any other state\index{State} than the input it receives upon creation.

We are only interested in \textbf{what} a process does, not \textbf{how} it does that. We treat processes as black boxes that receive an input, perform an action and eventually produce an output. Our point of interest is the composition of processes and the definition of the semantics of the resulting processes, which we give in denotational form.

\section{Syntax\index{Syntax}}
\label{chp:syntax}
Let $\mathcal{P}$ be the set of processes\index{Process} and let $P, Q, R \in \mathcal{P}$ be processes. Then syntactically correct processes can be formed by using the following construction rules:
\begin{itemize}
  \item $\choice{R}{P}{Q}$ \hspace*{2.3em} (Choice)\index{Syntax!Choice}
  \item $\parallel{R}{P}{Q}$ \hspace*{2.8em} (Parallel)\index{Syntax!Parallel}
  \item $\sequence{P}{Q}$ \hspace*{4.8em} (Sequence)\index{Syntax!Sequence}
  \item $\repetition{R}{P}$ \hspace*{4.4em} (Repetition)\index{Syntax!Repetition}
\end{itemize}


% ==============================================================================
% static semantics
% ==============================================================================

\section{Static Semantics\index{Semantics!Static}}
\label{chp:static_semantics}
In this section, we define the static semantics of the process combinators\index{Process!Combinator} in our calculus\index{Calculus}. We define what a \textbf{basic}\index{Process!Basic} and what a \textbf{composed}\index{Process!Composed} process is and we equip processes with a type signature\index{Type!Signature}. The \textbf{type signatures} of processes are used to determine whether they can be composed using a certain combinator to yield a valid process with respect to their types.

As before, let $\mathcal{P}$ be the set of processes. A basic\index{Process!Basic} process is a process that performs a basic operation an is no further divisible insofar, that it does not involve any of the introduced process combinators. A composed\index{Process!Composed} process is a process that is composed of other processes, using one or many process combinators.

Intuitively, we call values that we present to processes their \textbf{input}\index{Input} and values produced by processes their \textbf{output}\index{Output}.

\begin{definition}[Types\index{Type}]
Let $\mathcal{T}$ be the set of types and $T_i$ a type for every $i$ with
\begin{equation*}
  \mathcal{T} = \bigcup_i \left\{ T_i \right\}
\end{equation*}
Furthermore, let lowercase letters, e.g. $a, b, c$ be type variables\index{Type!Variable}, i.e. placeholders for types.

\hfill\qedsymbol
\end{definition}

A type represents a set of values, all of which share some common property and therefore, are attributed to the same type. Examples include the type of natural numbers $T_\mathbb{N}$, commonly referred to as \textit{integers}, and the type of boolean values $T_{Boolean} = \left\{False, True \right\}$.

\begin{definition}[Undefined values\index{Undefined Value}]
For every $T_i \in \mathcal{T}$, let $\bot_i \in T_i$ be the undefined value of type $T_i$ that is distinguishable from every other value of type $T_i$. Every type implicitly contains an undefined value, even if not explicitly mentioned\footnote{That means that the type of natural numbers $T_\mathbb{N}$ contains the undefined value $\bot_\mathbb{N}$ and the type $T_{Boolean}$ actually contains the three values $False$, $True$ and $\bot_{Boolean}$.}. When talking about the undefined value $\bot$, we omit its type and only mention it explicitly when necessary.

\hfill\qedsymbol
\end{definition}


% ------------------------------------------------------------------------------
% type signature
% ------------------------------------------------------------------------------
\begin{definition}[Type signature\index{Type!Signature}]
\label{def:type_signature}
The type signature of a process states of which types the input and the output of the respective process are. Let $\rho \colon \mathcal{P} \to \mathcal{T} \times \mathcal{T}$ be the function that assigns a type signature to processes.

\hfill\qedsymbol
\end{definition}

% ------------------------------------------------------------------------------
% identity and error process
% ------------------------------------------------------------------------------

\begin{definition}[Identity\index{Process!Identity} and error\index{Process!Error} process]
\label{def:static_id_err}
Let $Id \in \mathcal{P}$ be the identity process, i.e. the process that outputs its input, and let $Err \in \mathcal{P}$ be the error process, i.e. the process that always returns the undefined value.

\hfill\qedsymbol
\end{definition}

To be precise, there is an identity process $Id_i$ with $\typesignature{Id_i}{T_i}{T_i}$ and an error process $Err_i$ with $\typesignature{Err_i}{T_i}{T_i}$ for every $T_i \in \mathcal{T}$. For simplicity, we refer to them as $Id$ and $Err$, but keep in mind that for every type, there is a specific identity and error process. Let their type signatures be $\typesignature{Id}{a}{a}$ and $\typesignature{Err}{a}{a}$, where $a$ is a type variable resembling the fact that $Id$ and $Err$ represent a family of processes.

% ------------------------------------------------------------------------------
% processes as predicates
% ------------------------------------------------------------------------------
In the following, we also call processes $B \in \mathcal{P}$ with output type $T_{Boolean}$, i.e. with the property $\rho \left( B \right) = \left( a, T_{Boolean} \right)$ \textit{predicates}\index{Predicate}. They are used to express deterministic choice between processes and termination of repetition.

% ------------------------------------------------------------------------------
% choice
% ------------------------------------------------------------------------------
\begin{definition}[Static semantics of choice composition\index{Static Semantics!Choice}]
\label{def:static_choice}
Let $B$ be a predicate and let $P, Q \in \mathcal{P}$ be processes and let $a, b$ be type variables. Then the composition $\choice{B}{Q}{P}$ is valid iff
\begin{eqnarray*}
  \rho \left( B \right) & = & \left( a, T_{Boolean} \right) \\
  \rho \left( P \right) & = & \left( a, b \right) \\
  \rho \left( Q \right) & = & \left( a, b \right)
\end{eqnarray*}
and invalid otherwise. The type signature of the resulting process is $\typesignature{\choice{B}{P}{Q}}{a}{b}$.

\hfill\qedsymbol
\end{definition}



% ------------------------------------------------------------------------------
% parallel
% ------------------------------------------------------------------------------
\begin{definition}[Static semantics of parallel composition\index{Static Semantics!Parallel}]
\label{def:static_parallel}
Let $P, Q, R \in \mathcal{P}$ be processes and let $a, b, c, d$ be type variables. Then the composition $\parallel{R}{P}{Q}$ is valid iff
\begin{eqnarray*}
  \rho \left( P \right) & = & \left( a, c \right) \\
  \rho \left( Q \right) & = & \left( a, d \right) \\
  \rho \left( R \right) & = & \left( \left( c, d \right), b \right)
\end{eqnarray*}
and invalid otherwise. The type signature of the resulting process is $\typesignature{\parallel{R}{P}{Q}}{a}{b}$.

\hfill\qedsymbol
\end{definition}


% ------------------------------------------------------------------------------
% sequence
% ------------------------------------------------------------------------------
\begin{definition}[Static semantics of sequential composition\index{Static Semantics!Sequence}]
\label{def:static_sequence}
Let $P, Q \in \mathcal{P}$ be processes and let $a, b, c$ be type variables. Then the composition $\sequence{P}{Q}$ is valid iff
\begin{eqnarray*}
  \rho \left( P \right) & = & \left( a, c \right) \\
  \rho \left( Q \right) & = & \left( c, b \right)
\end{eqnarray*}
and invalid otherwise. The type signature of the resulting process is $\typesignature{\sequence{P}{Q}}{a}{b}$.

\hfill\qedsymbol
\end{definition}


% ------------------------------------------------------------------------------
% repetition
% ------------------------------------------------------------------------------
\begin{definition}[Static semantics of repetition\index{Static Semantics!Repetition}]
\label{def:static_repetition}
Let $B$ be a predicate, let $P \in \mathcal{P}$ be a process and let $a$ be a type variable. Then the composition $\repetition{B}{P}$ is valid iff
\begin{eqnarray*}
  \rho \left( B \right) & = & \left( a, T_{Boolean} \right) \\
  \rho \left( P \right) & = & \left( a, a \right) \\
\end{eqnarray*}
and invalid otherwise. The type signature of the resulting process is $\typesignature{\repetition{B}{P}}{a}{a}$.

\hfill\qedsymbol
\end{definition}



% ==============================================================================
% semantics
% ==============================================================================
\clearpage
\section{Semantics\index{Semantics}}
\label{chp:semantics}
In \chpref{chp:syntax} we have introduced the syntactical rules for process construction and in \chpref{chp:static_semantics} we have introduced static semantics of process composition. In this section, we define the semantics of processes by giving them a denotation as it is common in the design of programming languages \cite{DenSem}.

For the definition of process semantics, we introduce the polymorphic function 
\begin{equation*}
  sem \colon \mathcal{P} \to a \to b.
\end{equation*}
$sem$ takes a process $P \in \mathcal{P}$ and returns its meaning, i.e. the function computed by $P$. We write $\sem{P}$ for that. Since $\sem{P}$ is a function, it can be applied to a concrete input value $x$ for $P$, written as $\sem{P} \left( x \right)$. The expression $\sem{P} \left( x \right)$ is valid iff $x$ is of appropriate type to be used as input for $P$.

% ------------------------------------------------------------------------------
% atomic process
% ------------------------------------------------------------------------------
For every basic process, there is a function that is intrinsic to that particular process: for a basic process $P \in \mathcal{P}$, let its intrinsic function\index{Intrinsic Function} be called $f_P$. Composed processes do not involve an intrinsic function and the notation $f_P$ does not make sense if $P$ is not a basic process. The semantics of a basic\index{Process!Basic} process $P \in \mathcal{P}$ is directly given by its intrinsic function.
\begin{definition}[Semantics of a basic process\index{Semantics!Basic Process}]
  \label{def:sem_atomic}
  Let $P$ be a basic process, then the semantics $\sem{P}$ is given by $P$'s intrinsic function $f_P$:
  \begin{equation*}
    \label{eqn:sem_atomic}
    \sem{P} = f_P.
  \end{equation*}
  \vspace*{-0.5em}
  \hfill\qedsymbol
\end{definition}

\begin{example}[Semantics of basic processes]
  \label{exp:sem_atomic}
  Let $\sigma \colon T_\mathbb{N} \to T_\mathbb{N}, x \mapsto x+1$ be the function that is intrinsic to the basic process $S \in \mathcal{P}$. Then the semantics $\sem{S}$ of $S$ is given by its intrinsic function $\sigma$:
  \begin{equation*}
    \sem{S} = \sigma = x \mapsto x+1
  \end{equation*}
  Let $\delta \colon T_\mathbb{N} \to T_\mathbb{N}, x \mapsto 2x$ be the function that is intrinsic to the basic process $D \in \mathcal{P}$. Then the semantics $\sem{D}$ of $D$ is given by its intrinsic function $\delta$:
  \begin{equation*}
    \sem{D} = \delta = x \mapsto 2x
  \end{equation*}
\end{example}

% ------------------------------------------------------------------------------
% id
% ------------------------------------------------------------------------------
\begin{definition}[Semantics of the identity process\index{Semantics!Identity Process}\index{Process!Identity}\index{Identity Process}]
\label{def:sem_id}
Let $Id \in \mathcal{P}$ be the identity process, i.e. the process that always outputs its input. Its semantics is:
\begin{equation*}
  \label{eqn:sem_id}
  \sem{Id} = x \mapsto x.
\end{equation*}
  \vspace*{-0.5em}
  \hfill\qedsymbol
\end{definition}


The semantics of the error process $Err$, no matter which input it receives, is always the undefined value $\bot$ of the respective type. 
\begin{definition}[Semantics of the error process\index{Semantics!Error Process}\index{Process!Error}\index{Error Process}]
\label{def:sem_err}
Let $Err \in \mathcal{P}$ be the error process, i.e. the process that always outputs the undefined value\index{Undefined Value} $\bot$. Its semantics is:
  \begin{equation*}
    \label{eqn:sem_error}
    \sem{Err} = x \mapsto \bot.
  \end{equation*}
  \vspace*{-0.5em}
  \hfill\qedsymbol
\end{definition}

% ------------------------------------------------------------------------------
% composed processes
% ------------------------------------------------------------------------------
The semantics of composed\index{Process!Composed} processes is determined by their structure, i.e. the used process combinator, and the semantics of the involved sub-processes\index{Sub-process}. The semantics of processes created by choice composition, parallel composition and sequential composition is defined inductively. The semantics of processes created by repetition composition is given by a recursive equation.

\begin{definition}[Sub-process\index{Sub-process}]
Let $P \in \mathcal{P}$ be a composed process. Then the processes $P$ is composed of are called the sub-processes of $P$.

\hfill\qedsymbol
\end{definition}

% ------------------------------------------------------------------------------
% choice
% ------------------------------------------------------------------------------
\begin{definition}[Shorthand notation for predicate processes]
Let $B$ be a predicate\index{Predicate}. Then $B \left( x \right)$ is a shorthand notion for $\sem{B} \left( x \right) = True$ and $\overline{B \left( x \right)}$ is a shorthand notation for $\sem{B} \left( x \right) = False$. For non-predicates this notation is not defined.

\hfill\qedsymbol
\end{definition}


The semantics of a process with the structure of choice composition is that of one of its sub-processes. A predicate is employed to inspect the input and decide which of the sub-processes determines the semantics.
\begin{definition}[Semantics of choice composition\index{Semantics!Choice}]
\label{def:sem_choice}
Let $B$ be a predicate, let $P, Q \in \mathcal{P}$ be processes and let them be composed using choice composition, i.e. $\choice{B}{P}{Q}$. Then, based on $\sem{B}$, the resulting process behaves either like $P$ or $Q$. The semantics of $\choice{B}{P}{Q}$ is given by:
  \begin{equation*}
    \label{eqn:sem_choice}
    \sem{\choice{B}{P}{Q}} = x \mapsto \left\{ \begin{array}{ll}
      \sem{P}\left(x\right) & \text{if } B \left( x \right) \\
      \sem{Q}\left(x\right) & \text{if } \overline{B \left( x \right)} \\
      \bot & \text{otherwise}.
    \end{array}\right.
  \end{equation*}
  
  \hfill\qedsymbol
\end{definition}

\begin{example}[Semantics of choice composition]
\label{exp:sem_chice}
Consider the definitions of $S$ and $D$ from \expref{exp:sem_atomic}. Furthermore, let $Even \in \mathcal{P}$ with $\typesignature{Even}{\mathbb{N}}{T_{Bool}}$ be the process that outputs $True$ if its input is an even number and $False$ otherwise. If we compose $S$ and $D$, guarded by $Even$, using choice composition, the semantics of the resulting process is as follows:
  \begin{equation*}
    \sem{\choice{Even}{S}{D}} = x \mapsto \left\{\begin{array}{ll}
      \sem{S}\left(x\right) & \text{if } x \text{ is even} \\
      \sem{D}\left(x\right) & \text{if } x \text{ is odd} \\
      \bot & \text{otherwise}
    \end{array}\right.
  \end{equation*}
\end{example}


% ------------------------------------------------------------------------------
% parallel
% ------------------------------------------------------------------------------
The semantics of a process with the structure of parallel composition is given by the semantics of its sub-processes, combined by a combinator process.
\begin{definition}[Semantics of parallel composition\index{Semantics!Parallel}]
\label{def:sem_parallel}
Let $P, Q, R \in \mathcal{P}$ be processes and let them be composed using parallel composition, i.e. $\parallel{R}{P}{Q}$. Then the semantics of the resulting process $\parallel{R}{P}{Q}$ is given by:
  \begin{equation*}
    \label{eqn:sem_parallel}
    \sem{\parallel{R}{P}{Q}} = x \mapsto \sem{R} \left( \sem{P} \left( x \right), \sem{Q} \left( x \right) \right).
  \end{equation*}
  \hfill\qedsymbol
\end{definition}


\begin{example}[Semantics of parallel composition]
\label{exp:sem_parallel}
Consider the definitions of $\sigma, \delta, S$ and $D$ from \expref{exp:sem_atomic}. Furthermore, let $sum \colon T_\mathbb{N} \times T_\mathbb{N} \to T_\mathbb{N}, \left( x, y \right) \mapsto x + y$ be the function that is intrinsic to the basic process $Sum \in \mathcal{P}$. Then the semantics of the process created by parallel composition of $S$ and $D$, combined with $Sum$, is: 
  \begin{eqnarray*}
    \sem{\parallel{Sum}{S}{D}} & = & x \mapsto \sem{Sum}{\left( \sem{S} \left( x \right), \sem{D} \left( x \right) \right)} \\
                               & = & x \mapsto \sem{Sum}{\left( \sigma\left(x\right), \delta\left(x\right) \right)} \\
                               & = & x \mapsto sum \left( \sigma\left(x\right), \delta\left(x\right) \right) \\
                               & = & x \mapsto sum \left( x+1, 2x \right) \\
                               & = & x \mapsto \left( x+1 \right) + \left( 2x \right) \\
                               & = & x \mapsto 3x + 1
  \end{eqnarray*}
\end{example}


% ------------------------------------------------------------------------------
% sequence
% ------------------------------------------------------------------------------
The semantics of a process with the structure of sequential composition is given by applying the function defined by its sub-processes sequentially. This can be expressed using function composition.
\begin{definition}[Semantics of sequential composition\index{Semantics!Sequence}]
\label{def:sem_sequence}
Let $P, Q \in \mathcal{P}$ be processes and let them be composed using sequential composition, i.e. $\sequence{P}{Q}$. Then the semantics of the resulting process is given by function composition:
  \begin{equation*}
    \label{eqn:sem_sequence}
    \sem{\sequence{P}{Q}} = \sem{Q} \circ \sem{P} = x \mapsto \sem{Q} \left( \sem{P} \left( x \right) \right)
  \end{equation*}
  \hfill\qedsymbol
\end{definition}


\begin{example}[Semantics of sequential composition]
\label{exp:sem_sequence}
Consider the definitions of $\sigma, \delta, S$ and $D$ from \expref{exp:sem_atomic} and use sequential composition to compose $D$ and $S$. Since function composition $\circ$ does not have commutative properties, $\sequence{S}{D}$ and $\sequence{D}{S}$ yield different results.
  \begin{eqnarray*}
    \sem{\sequence{S}{D}} & = & \sem{D} \circ \sem{S} \\
                          & = & \delta \circ \sigma \\
                          & = & x \mapsto \delta \left( \sigma \left( x \right) \right) \\
                          & = & x \mapsto 2 \left( x+1 \right) \\
                          & = & x \mapsto 2x+2
  \end{eqnarray*}
  \begin{eqnarray*}
    \sem{\sequence{D}{S}} & = & \sem{S} \circ \sem{D} \\
                          & = & \sigma \circ \delta \\
                          & = & x \mapsto \sigma \left( \delta \left( x \right) \right) \\
                          & = & x \mapsto 2x + 1
  \end{eqnarray*}
\end{example}


% ------------------------------------------------------------------------------
% repetition
% ------------------------------------------------------------------------------
The semantics of a process with the structure of repetition composition is given by a recursive equation. It involves repeated sequential composition of a process with itself. A predicate is employed to test the process's input for a property and determine the length of the process composition sequence.
\begin{definition}[Semantics of repetition composition\index{Semantics!Repetition}]
\label{def:sem_repetition}
Let $B$ be a predicate, let $P \in \mathcal{P}$ be processes and let them be composed using repetition composition, i.e. $\repetition{B}{P}$. Then the semantics of $\repetition{B}{P}$ is given by the following recursive equation:
  \begin{equation*}
    \label{eqn_sem_repetition}
    \sem{\repetition{B}{P}} = \sem{\choice{B}{\sequence{P}{\repetition{B}{P}}}{Id}}.
  \end{equation*}
  \hfill\qedsymbol
\end{definition}

The existence of a solution satisfying \defref{def:sem_repetition} is not immediately obvious. However, this is a well-known and well-studied problem. The theory of \textit{least fixed point semantics}\index{Least Fixed Point Semantics} from domain theory\index{Domain Theory} guarantees the existence of a solution and gives information about how to find it \cite{DenSem}.

\begin{example}[Semantics of repetition composition\index{Semantics!Repetition}]
\label{exp:sem_repetition}
Consider the definition of $\delta$ and $D$ from \expref{exp:sem_atomic} and let $B$ be a predicate that returns $True$ if its input is smaller than 1024 and $False$ if its input is equal to 1024 or bigger.
  \begin{eqnarray*}
  & & \sem{\repetition{B}{D}} \\
    & = & \sem{\choice{B}{\sequence{D}{\repetition{B}{D}}}{Id}} \\
    & = & x \mapsto \left\{ \begin{array}{ll} 
                              \sem{\sequence{D}{\repetition{B}{D}}} \left( x \right) & \text{if } x < 1024 \\
                              \sem{Id} \left( x \right)                              & \text{if } x \geq 1024 \\
                              \bot                                                   & \text{otherwise}
                            \end{array}
                    \right. \\
    & = & x \mapsto \left\{ \begin{array}{ll} 
                              \sem{\sequence{D}{\choice{B}{\sequence{D}{\repetition{B}{D}}}{Id}}} \left( x \right) & \text{if } x < 1024 \\
                              \sem{Id} \left( x \right)                              & \text{if } x \geq 1024 \\
                              \bot                                                   & \text{otherwise}
                            \end{array}
                    \right. \\
    & = & \ldots
  \end{eqnarray*}
$\repetition{B}{D}$ yields a process that takes its input, doubles it and passes it on recursively to itself as long as its input is smaller than 1024. If its input is bigger than 1024, it outputs its input.
\end{example}


% ------------------------------------------------------------------------------
% equivalence
% ------------------------------------------------------------------------------
\clearpage
\section{Semantic equivalence and substitution\index{Semantic Equivalence}\index{Substitution}}
\label{chp:semantic_equivalence}
Based on the definition of process semantics from \chpref{chp:semantics}, we have a tool at hand to reason about processes and perform transformations on them.

\begin{definition}[Equivalence of processes\index{Process!Equivalence}]
\label{def:process_equivalence}
Let $P, Q \in \mathcal{P}$ be processes. Then $P$ and $Q$ are equivalent iff their semantics are the same. We write $P \equiv Q$ to indicate equivalence of $P$ and $Q$.
  \begin{equation*}
    \label{eqn:equivalence}
    P \equiv Q \Leftrightarrow \sem{P} = \sem{Q}
  \end{equation*}
  \hfill\qedsymbol
\end{definition}


\begin{definition}[Process substitution\index{Process!Substitution}]
\label{def:process_substitution}
Let $P, Q, R \in \mathcal{P}$ be processes and let $Q$ be a sub-process of $P$. Then $P$ can be transformed into a new process by replacing one or more occurrence of $Q$ in $P$ with $R$. This is denoted by $P_{\left[ Q / R \right]}$.

\hfill\qedsymbol
\end{definition}


\begin{theorem}[Substitution with an equivalent process leaves semantics unchanged]
\label{thm:process_substitution}
Let $P, Q, R \in \mathcal{P}$ be processes and let $Q \equiv R$. Then, substituting one or more occurrences of $Q$ in $P$ with $R$ leaves $\sem{P}$ unchanged:
  \begin{equation*}
    Q \equiv R \rightarrow P \equiv P_{\left[ Q / R \right]}.
  \end{equation*}
  
  \hfill\qedsymbol
\end{theorem}


The property stated in \thmref{thm:process_substitution} allows us to transform processes into a different representation without altering their semantics and to optimise them according to some measurable property. E.g. one might want to find a more \enquote{simple} representation of a process or reduce its complexity by replacing sub-processes. A proof of \thmref{thm:process_substitution} can be found in \appref{chp:proofs}, proof \ref{prf:substitution}.

% ------------------------------------------------------------------------------
% inverse process
% ------------------------------------------------------------------------------
\begin{definition}[Inverse process\index{Process!Inverse}\index{Inverse Process}]
\label{def:inverse_process}
Let $P \in \mathcal{P}$ be a process. Then an inverse process $\overline{P}$ for $P$ is a process that satisfies the following property:
  \begin{equation*}
    \sequence{P}{\overline{P}} \equiv Id.
  \end{equation*}
  \hfill\qedsymbol
\end{definition}


As a final remark, we point out that, for a process $P$, an inverse process as defined in \defref{def:inverse_process} does \textbf{not} exist in general since not every computable function is invertible\footnote{Take, e.g. the function $f \colon x \mapsto x^2$. Clearly, $f$ is not injective since $f \left( x \right) = f \left( -x \right)$ and therefore not invertible. However, there is an inverse relation $\overline{f_r} \colon x \mapsto \left\{ \overline{x} \,|\, f \left( \overline{x} \right) = x \right\}$ that maps a value $x$ to the set of values that, if $f$ is applied to them, yield $x$.}.



% ==============================================================================
% laws
% ==============================================================================
\clearpage
\section{Laws\index{Laws}}
\label{chp:laws}
In \chpref{chp:semantic_equivalence}, \defref{def:process_equivalence}, we have defined under which circumstances two processes are equivalent. This definition, together with the definitions for process semantics given in \chpref{chp:semantics}, allows us to find a set of laws regarding process equivalence in our calculus that can be used by the user of the calculus to transform processes into different representations, satisfying his needs.

In the following, let $B$ be a predicate and let $C, P, Q, R \in \mathcal{P}$ be processes with suitable type signatures for composition in the respective cases. Proofs for the stated laws can be found in \appref{chp:proofs}.

\subsection{Associativity\index{Associativity}\index{Laws!Associativity}}
The associative property holds for sequential composition.
\begin{eqnarray*}
  \sequence{P}{\sequence{Q}{R}} & \equiv & \sequence{\sequence{P}{Q}}{R}
\end{eqnarray*}

\subsection{Distributivity\index{Distributivity}\index{Laws!Distributivity}}
Parallel and sequential composition distribute over choice composition. Also, sequence composition distributes over parallel composition.
\begin{eqnarray*}
  \parallel{C}{P}{\choice{B}{Q}{R}} & \equiv & \choice{B}{\parallel{C}{P}{Q}}{\parallel{C}{P}{R}} \\
  \sequence{P}{\choice{B}{Q}{R}} & \equiv & \choice{B}{\sequence{P}{Q}}{\sequence{P}{R}} \\
  \sequence{P}{\parallel{C}{Q}{R}} & \equiv & \parallel{C}{\sequence{P}{Q}}{\sequence{P}{R}}
\end{eqnarray*}

\subsection{Neutral elements\index{Neutral Element}\index{Laws!Neutral Element}}
The identity process $Id$ is both a left and right neutral element for sequential composition.
\begin{eqnarray*}
  \sequence{Id}{P} & \equiv & P \\
  \sequence{P}{Id} & \equiv & P
\end{eqnarray*}

\subsection{Idempotence\index{Idempotence}\index{Laws!Idempotence}}
The identity process $Id$ and the error process $Err$ are idempotent\footnote{Let $S$ be a set and $\ast \colon S \times S \to S$ be a binary operation on the elements of $S$. An element $x \in S$ is considered to be idempotent with respect to $\ast$ if $x \ast x = x$. If for all $x \in S$ the idempotence property $x \ast x = x$ holds, the operation $\ast$ is considered idempotent.} elements regarding sequential composition.
\begin{eqnarray*}
  \sequence{Id}{Id} & \equiv & Id \\
  \sequence{Err}{Err} & \equiv & Err \\
\end{eqnarray*}