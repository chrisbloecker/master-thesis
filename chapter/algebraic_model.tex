\chapter{An algebraic model}
In the previous chapter we have seen two examples of process calculi and how they work. Before we start to define our own calculus for distributed programming, we will develop a yet more abstract algebraic model, similar to the one presented in \cite{Hoare:2012:LPU:2368298.2368301}. We will use it to define the semantics of our calculus, which will implement the algebraic model.

First, let us define what we mean by \textit{process}: a process is (a piece of) a computer program. It can run concurrently with other processes. A process receives an input and produces an output. Essentially, a process can be seen as a function. % ToDo?

For now we are not interested in \textbf{what} a process does and \textbf{how} it does that. We will treat processes as black boxes that receive an input, perform an action and eventually produce an output. Our point of interest is the composability of processes and the algebraic structure we can define on them.

\section{Syntax}

Let $\mathcal{P}$ be the set of processes and let $P, Q \in \mathcal{P}$ be syntactically correct processes. Then
\begin{itemize}
  \item $P \vee Q$
  \vspace*{-0.25em}
  \item $P \circ Q$
  \vspace*{-0.25em}
  \item $P \,|\, Q$
  \vspace*{-0.25em}
  \item $P^*$
  \vspace*{-0.25em}
  \item $\left( P \right)$
\end{itemize}
form syntactically correct processes.

The syntax can also be expressed in form of a grammar $G = \left( N, T, R, P \right)$, where $N = \left\{ P \right\}$ is the set of non-terminals, $T = \left\{ ??? \right\}$ is the set of terminals, $R$ is the set of production rules and $P$ is the start symbol \cite{Hopcroft:2006:IAT:1196416} with % ToDo!
\begin{eqnarray*}
  R \colon \quad P & \to & P \vee P \\
    & | & P \circ P \\
    & | & P \,|\, P \\
    & | & P^* \\
    & | & \left( P \right).
\end{eqnarray*}
Note that $G$ is context-free since it involves a Dyck language \cite{}.

\section{Semantics}
The operator precedence in our algebra is a follows: $*$ binds strongest, the next weaker operator is $\circ$, followed by $|$ and in the end $\vee$. Parentheses, as described in the syntax, might be necessary to express the desired composition.

Let $\mathcal{T} = \left\{ \bot \right\} \, \cup \, \bigcup_i \left\{ t_i \right\}$ be the set of types and let $t_i \in \mathcal{T}$ be types for every $i$. Let $\rho \colon \mathcal{P} \to \mathcal{T} \times \mathcal{T}$ be a function that assigns a type signature to a process. Furthermore, let $Id$ be the process that simply outputs its input, i.e. the identity process and let $Err \in \mathcal{P}$ the error process, i.e. the process that always fails. Let their type signatures be $\rho \left( Id \right) = \left( t_*, t_* \right)$ and $\rho \left( Err \right) = \left( t_*, \bot \right)$, where $t_*$ is a polymorphic type and $\bot$ is the undefined type. $Id$ and $Err$ will match their concrete input type to a specific $t_i \in \mathcal{T}$, which is determined by the context.

Let $\vee \colon \mathcal{P} \times \mathcal{P} \to \mathcal{P}$ be the choice between two processes. $P \vee Q$ yields a process that either behaves like $P$ or $Q$. $\mathcal{P}$ is closed under $\vee$ since $\vee$ selects between $P$ and $Q$ and $P, Q \in \mathcal{P}$. In order to compose processes with $\vee$, their type signatures have to match. Let
\begin{eqnarray*}
  \rho \left( P \right) & = & \left( t_i, t_j \right) \\
  \rho \left( Q \right) & = & \left( t_k, t_l \right).
\end{eqnarray*}
Then $P \vee Q$ yields a valid process with $\rho \left( P \vee Q \right) = \left( t_i, t_j \right)$ iff $t_i = t_k \wedge t_j = t_l$ and $Err$ otherwise.

Let $\circ \colon \mathcal{P} \times \mathcal{P} \to \mathcal{P}$ be the sequential composition of two processes. $Q \circ P$ means: first, execute $P$, then, execute $Q$. Clearly, $\mathcal{P}$ is closed under $\circ$ since the result is a process that lies in $\mathcal{P}$ by definition. Note that in order to compose processes with $\circ$, their type signatures have to match. Suppose, we want to compose processes $P$ and $Q$ using $\circ$, i.e. $Q \circ P$ and
\begin{eqnarray*}
  \rho \left( P \right) & = & \left( t_0, t_i \right) \\
  \rho \left( Q \right) & = & \left( t_j, t_1 \right).
\end{eqnarray*}
Then the composition $Q \circ P$ yields a valid process with $\rho \left( Q \circ P \right) = \left( t_0, t_1 \right)$ iff $t_i = t_j$, and results in $Err$ otherwise. 

Let $| \colon \mathcal{P} \times \mathcal{P} \to \mathcal{P}$ be the parallel composition of processes, i.e. running processes $P$ and $Q$ concurrently. Then $P \,|\, Q$ will creates a process that executes $P$ and $Q$ concurrently. Suppose, we want to compose $P$ and $Q$ using $|$, i.e. $P \,|\, Q$ and
\begin{eqnarray*}
  \rho \left( P \right) & = & \left( t_i, t_j \right) \\
  \rho \left( Q \right) & = & \left( t_k, t_l \right).
\end{eqnarray*}
Then the composition $P \,|\, Q$ yields a valid process with $\rho \left( P \,|\, Q \right) = \left( t_i, t_j \right)$ iff $t_i = t_k \wedge t_j = t_l$, i.e. $\rho \left( P \right) = \rho \left( Q \right)$. Furthermore, we require $t_j$ to form a semigroup so we can combine the results of processes $P$ and $Q$ together into a single element of $t_j$. Note that $P \,|\, Q = Err$ iff $\rho \left( P \right) \neq \rho \left( Q \right)$.

Let $* \colon \mathcal{P} \to \mathcal{P}$ be the Kleene star that can be used to express repetition, i.e. executing a process several times. $P^*$ creates a process that runs $P$ zero or more times and can also be expressed recursively as $P^* = Id \vee P^* \circ P$. Let $\rho \left( P \right) = \left( t_i, t_j \right)$, then $P^*$ is valid and $\rho \left( P^* \right) = \left( t_i, t_i \right)$ iff $t_i = t_j$ and $Err$ otherwise.

Parentheses will be neccessary in some situations to overcome operator precedence and to express the desired composition of processes. They have no effect of the semantics of a single process, i.e. for a process $P$, let $R = \left( P \right)$, then $P = R$ and $\rho \left( P \right) = \rho \left( R \right)$.

\section{Laws}
Based on the defined semantics of our process algebra, we can find a set of laws that allow us to transform process expressions into an equivalent representation. In the following, let $P, Q, R \in \mathcal{P}$ and assume that their types are suitable for the desired composition.

\subsection{Associativity}
\begin{equation*}
\begin{array}{rcccl}
  \left( R \vee Q \right) \vee P & = & R \vee Q \vee P & = & R \vee \left( Q \vee P \right) \\
  \left( R \,|\, Q \right) |\, P & = & R \,|\, Q \,|\, P & = & R \,| \left( Q \,|\, P \right) \\
  \left( R \circ Q \right) \circ P & = & R \circ Q \circ P & = & R \circ \left( Q \circ P \right)
\end{array}
\end{equation*}

\subsection{Distributivity}
\begin{eqnarray*}
  P \vee \left( Q \vee R \right) & = & \left( P \vee Q \right) \vee \left( P \vee R \right) \\
  P \,| \left( Q \vee R \right) & = & \left( P \,|\, Q \right) \vee \left( P \,|\, R \right) \\
  P \circ \left( Q \vee R \right) & = & \left( P \circ Q \right) \vee \left( P \circ R \right) \\
  & & \\
  P \circ \left( Q \,|\, R \right) & = & \left( P \circ Q \right) | \left( P \circ R \right)
\end{eqnarray*}

\subsection{Commutativity}
\begin{eqnarray*}
  P \vee Q & = & Q \vee P \\
  P \,|\, Q & = & Q \,|\, P
\end{eqnarray*}

\subsection{Idempotence}
\begin{eqnarray*}
  P \vee P & = & P \\
  Id \circ Id & = & Id \\
  Err \circ Err & = & Err \\
  (P^*)^* & = & P^*
\end{eqnarray*}

\subsection{Neutral elements}
\begin{eqnarray*}
  P \circ Id & = & P \\
  Id \circ P & = & P
\end{eqnarray*}

\subsection{Further laws}
\begin{eqnarray*}
  Err \circ P & = & Err \\
  P \circ Err & = & Err
\end{eqnarray*}