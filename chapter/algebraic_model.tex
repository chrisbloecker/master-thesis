\chapter{An algebraic model}
\label{chp:algebraic_model}
In the previous chapter we have seen two examples of process calculi and how they work. Before we start to define our own calculus for distributed programming, we develop a yet more abstract algebraic model, similar to the one presented in \cite{Hoare:2012:LPU:2368298.2368301}. We use it to define the semantics of our calculus, which implements the algebraic model.

First, let us define what we mean by \textit{process}: a process is (a piece of) a computer program. It can run concurrently with other processes. A process receives an input and produces an output. Essentially, a process can be seen as a function and just like that, it is pure and free of side-effects, it does not have any other state than the input it receives upon creation.

For now we are not interested in \textbf{what} a process does or \textbf{how} it does that. We treat processes as black boxes that receive an input, perform an action and eventually produce an output. Our point of interest is the composability of processes and the algebraic structure we can define on them.

\section{Syntax}
\label{chp:syntax}
Let $\mathcal{P}$ be the set of processes and let $P, Q \in \mathcal{P}$ be basic processes. $\mathcal{P}$ represents the set of computable functions in form of processes and hence is countable. Then syntactically correct processes can be formed inductively by:
\begin{itemize}
  \item $P \vee Q$ \hspace*{2em} (Choice)
  \vspace*{-0.25em}
  \item $P \,|\, Q$ \hspace*{2.5em} (Parallel)
  \vspace*{-0.25em}
  \item $P \circ Q$ \hspace*{2.2em} (Sequence)
  \vspace*{-0.25em}
  \item $P^*$ \hspace*{3.5em} (Repetition)
  \vspace*{-0.25em}
  \item $\left( P \right)$.
\end{itemize}

%The syntax can also be expressed in form of a grammar $G = \left( N, T, R, P \right)$, where $N = \left\{ P \right\}$ is the set of non-terminals, $T = \mathcal{P}$ is the set of terminals, $R$ is the set of production rules and $P$ is the start symbol \cite{Hopcroft:2006:IAT:1196416} with % ToDo!
%\begin{eqnarray*}
%  R \colon \quad P & \to & P \vee P \\
%    & | & P \,|\, P \\
%    & | & P \circ P \\
%    & | & P^* \\
%    & | & \left( P \right) \\
%    & | & p \quad\quad\quad\quad \text{for every } p \in \mathcal{P}
%\end{eqnarray*}
%Note that $G$ is context-free-like since it involves a Dyck language \cite{} and a countable set of terminals.




% ==============================================================================
% static semantics
% ==============================================================================

\section{Static Semantics}
\label{chp:static_semantics}
In this chapter we define the static semantics of our algebraic model. We define what a basic and what a composed process is and we equip processes with a type signature. The type signature of a process determines if the process can be used for composition with other processes, depending on the selected combinator.

As before, let $\mathcal{P}$ be the countable set of processes. A basic, or atomic, process is a process that performs an atomic operation and is no further divisable. A composed process is a process that is composed of other processes, using one or many of the introduced process combinators $\vee, |, \circ$ and $*$.

\begin{definition}[Operator precedence]
\label{def:operator_precedence}
The operator precedence in our algebra is as follows: $*$ binds strongest, the next weaker binding operator is $\circ$, followed by $|$ and in the end $\vee$. Parentheses, as described in the syntax, might be necessary to express the desired composition.

\hfill\qedsymbol
\end{definition}

\begin{definition}[Types]
Let $\mathcal{T}$ be the set of types with
\begin{equation}
  \mathcal{T} = \left\{ \bot \right\} \, \cup \, \bigcup_{i \in \mathbb{N}_0} \left\{ T_i \right\}
\end{equation}
and let $T_i \in \mathcal{T}$ be types for every $i$. Furthermore, let $\bot$ be the undefined type that is identified with its only value $\bot$.

\hfill\qedsymbol
\end{definition}

% ------------------------------------------------------------------------------
% type signature
% ------------------------------------------------------------------------------
\begin{definition}[Type signature]
\label{def:type_signature}
The type signature of a process states types of which the input value and the output value of a process are.

\hfill\qedsymbol
\end{definition}

In the following, let $\rho \colon \mathcal{P} \to \mathcal{T} \times \mathcal{T}$ be a function that assigns a type signature to a process. Furthermore, let $Id \in \mathcal{P}$ be the process that simply outputs its input, i.e. the identity process, and let $Err \in \mathcal{P}$ the error process, i.e. the process that always fails. To be precise, there is an identity process $Id_i$ with $\rho \left( Id_i \right) = \left( T_i, T_i \right)$ and an error process $Err_i$ with $\rho \left( Err_i \right) = \left( T_i, \bot \right)$ for every $T_i \in \mathcal{T}$. For simplicity, we will just refer to them as $Id$ and $Err$, but keep in mind that for every type, there is a specific identity and error process. Let their type signatures be $\rho \left( Id \right) = \left( a, a \right)$ and $\rho \left( Err \right) = \left( a, \bot \right)$, where $a$ is a type variable which resembles the fact that $Id$ and $Err$ are representing a family of processes.

% ------------------------------------------------------------------------------
% strictness
% ------------------------------------------------------------------------------
\begin{definition}[Strictness of processes]
\label{def:process_strictness}
Processes are strict in the undefined value. If the undefined value $\bot$ is presented to a process as its input, then the output of the process is the undefined value $\bot$. Processes also give the undefined value as output whenever they receive an input value that lies outside their domain.

\hfill\qedsymbol
\end{definition}


% ------------------------------------------------------------------------------
% choice
% ------------------------------------------------------------------------------
\begin{definition}[Static semantics of the choice operator]
\label{def:static_choice}
Let $\vee \colon \mathcal{P} \times \mathcal{P} \to \mathcal{P}$ be the non-deterministic choice between two processes that will be made by an optimistic oracle\footnote{Our oracle will make this choice optimistically and never select the error process if it is involved.}. In order to compose processes with $\vee$, their type signatures have to match. Let $P, Q \in \mathcal{P}$ be processes and let $T_i, T_j, T_k, T_l \in \mathcal{T}$ be types. The type signatures of $P$ and $Q$ are given by $\rho$:
\begin{eqnarray}
  \rho \left( P \right) & = & \left( T_i, T_j \right) \\
  \rho \left( Q \right) & = & \left( T_k, T_l \right)
\end{eqnarray}
Then $P \vee Q$ yields a valid process with $\rho \left( P \vee Q \right) = \left( T_i, T_j \right)$ iff $T_i = T_k \wedge T_j = T_l$, otherwise the composition is invalid. $P \vee Q$ yields a process that either behaves like $P$ or $Q$, depending on the oracle's choice. Depending on the point of view, we could also say that $P \vee Q$ behaves like $P$ and $Q$ at the same time.

\hfill\qedsymbol
\end{definition}


% ------------------------------------------------------------------------------
% parallel
% ------------------------------------------------------------------------------
\begin{definition}[Static semantics of the parallel operator]
Let $| \colon \mathcal{P} \times \mathcal{P} \to \mathcal{P}$ be the parallel composition of processes. Let $P, Q \in \mathcal{P}$ be processes and let $T_i, T_j, T_k, T_l \in \mathcal{T}$ be types. The type signatures of $P$ and $Q$ are given by $\rho$:
\begin{eqnarray*}
  \rho \left( P \right) & = & \left( T_i, T_j \right) \\
  \rho \left( Q \right) & = & \left( T_k, T_l \right).
\end{eqnarray*}
Then the composition $P \,|\, Q$ yields a valid process with the signature $\rho \left( P \,|\, Q \right) = \left( T_i, T_j \right)$ iff $T_i = T_k \wedge T_j = T_l$, i.e. $\rho \left( P \right) = \rho \left( Q \right)$, and is invalid otherwise.

\hfill\qedsymbol
\end{definition}


% ------------------------------------------------------------------------------
% sequence
% ------------------------------------------------------------------------------
\begin{definition}[Static semantics of the sequence operator]
\label{def:static_sequence}
Let $\circ \colon \mathcal{P} \times \mathcal{P} \to \mathcal{P}$ be the sequential composition of two processes. Let $P, Q \in \mathcal{P}$ be processes and let $T_i, T_j, T_k, T_l \in \mathcal{T}$ be types. The type signatures of $P$ and $Q$ are given by $\rho$: 
\begin{eqnarray*}
  \rho \left( P \right) & = & \left( T_i, T_j \right) \\
  \rho \left( Q \right) & = & \left( T_k, T_l \right).
\end{eqnarray*}
Then the composition $Q \circ P$ yields a valid process with $\rho \left( Q \circ P \right) = \left( T_i, T_l \right)$ iff $T_j = T_k$, and is invalid otherwise. By $Q \circ P$, we mean running $P$ first and then $Q$.

\hfill\qedsymbol
\end{definition}


% ------------------------------------------------------------------------------
% kleene
% ------------------------------------------------------------------------------
\begin{definition}[Static semantics of the \textsc{Kleene} star]
\label{def:static_kleene}
Let $* \colon \mathcal{P} \to \mathcal{P}$ be the \textsc{Kleene} star that can be used to express repetition, i.e. executing a process several times. $P^*$ creates a process that runs $P$ zero or more times and can also be expressed recursively as $P^* = Id \vee P^* \circ P$. Let $\rho \left( P \right) = \left( T_i, T_j \right)$, then $P^*$ is valid and $\rho \left( P^* \right) = \left( T_i, T_i \right)$ iff $T_i = T_j$ and invalid otherwise. Like for the choice operator $\vee$, an oracle selects the concrete, \textbf{finite} number of repetitions of a process $P^*$, thus leading to non-determinism.

\hfill\qedsymbol
\end{definition}



% ==============================================================================
% semantics
% ==============================================================================

\section{Semantics}
\label{chp:semantics}
In \chpref{chp:static_semantics} we have introduced static semantics and construction rules for \textsc{Hive} processes. In this chapter, we will give an inductive definition of their semantics. In order to make this definition accurately, we need to introduce some terms first: \textit{argument} of a process, \textit{result} of a process and \textit{running} a process.

The argument $a$ of a process $P$ is the value that is presented to $P$ and that will be transformed into a result value $r$ by $P$. The type of $a$ needs to match $P$'s input type, which is given by $\rho$. The relationship of $P$, $a$ and $r$ can be described using \textsc{Hoare} triples from \textsc{Hoare} logic \cite{}. In \textsc{Hoare} logic, $a$ is a precondition, $r$ is a postcondition and $P$ is a command. The relationship between the three is given by $\left\{ a \right\} P \left\{ r \right\}$ and means that $P$ transform the program's state from a situation where $a$ holds to a situation where $r$ holds. To be precise, in \textsc{Hoare} logic both $a$ and $r$ are formulas in predicate logic. They are used to give certain assurances over the current program state. However, we can interpret $a$ and $r$ in a compatible way: Let $s$ be the state of the program, then the condition $a$ would mean that $s$ has to be exactly $a$.

By running a process $P$, we mean presenting an argument $a$ to the process, applying its intrinsic function to $a$ and receiving a result value $r$, i.e. $\left\{ a \right\} P \left\{ r \right\}$. The goal of this chapter is to define the intrinsic function of a process which is exactly its semantics.

For the definition of process semantics, we introduce a function $sem \colon \mathcal{P} \to 2^{\left( \mathcal{T} \to \mathcal{T} \right)}$ that takes a process and associated it with its meaning, i.e. with the \textbf{set} of functions it calculates. Since the choice operator $\vee$ and the \textsc{Kleene} star $*$ introduce non-determinism, every process that involves a choice between processes or the \textsc{Kleene} star will have more than one possible outcome, which we resemble in the definition of $sem$.


% ------------------------------------------------------------------------------
% atomic process
% ------------------------------------------------------------------------------
The semantics of an atomic process $P \in \mathcal{P}$ is directly given by the function it calculates. For an atomic process $P$, let $f_P$ be its intrinsic function.
\begin{definition}[Semantics of atomic processes]
  \label{def:sem_atomic}
  Let $P$ be an atomic process, then its semantics $sem \left\langle P \right\rangle$ is given by its intrinsic function $f_P$.
  \begin{equation}
    \label{eqn:sem_atomic}
    sem \left\langle P \right\rangle = \left\{ f_P \right\}
  \end{equation}
  \hfill\qedsymbol
\end{definition}

\begin{definition}[Semantics of the identity process]
\label{def:sem_id}
Let $Id \in \mathcal{P}$ be the identity process, i.e. the process that always returns its argument.
\begin{equation}
  \label{eqn:sem_id}
  sem \left\langle Id \right\rangle = \left\{ x \mapsto x \right\}
\end{equation}
\hfill\qedsymbol
\end{definition}


\begin{example}[Semantics of an atomic process]
  \label{exp:sem_atomic}
  Let $\sigma \colon x \mapsto x+1 \subset \mathbb{N} \times \mathbb{N}$ be the function that is intrinsic to the atomic process $S \in \mathcal{P}$. Then the semantics $sem \left\langle S \right\rangle$ of $S$ is given by its intrinsic function $\sigma$:
  \begin{equation}
    sem \left\langle S \right\rangle = \left\{ \sigma \right\} = \left\{ x \mapsto x+1 \right\}
  \end{equation}
  Let $\delta \colon x \mapsto 2x \subset \mathbb{N} \times \mathbb{N}$ be the function that is intrinsic to the atomic process $D \in \mathcal{P}$. Then the semantics $sem \left\langle D \right\rangle$ of $D$ is given by its intrinsic function $\delta$:
  \begin{equation}
    sem \left\langle D \right\rangle = \left\{ \delta \right\} = \left\{ x \mapsto 2x \right\}
  \end{equation}
  \hfill\qedsymbol
\end{example}

The error process $Err$ presents a special case here: as stated in \chpref{chp:static_semantics}, its output type is undefined. Hence, its result is always the undefined value, which is identified with the undefined type:
\begin{definition}[Semantics of the error process]
\label{def:sem_err}
Let $Err \in \mathcal{P}$ be the error process, i.e. the process that always returns the undefined value $\bot$.
  \begin{equation}
    \label{eqn:sem_error}
    sem \left\langle Err \right\rangle = \left\{ \bot \right\}.
  \end{equation}
  \hfill\qedsymbol
\end{definition}

% ------------------------------------------------------------------------------
% composed processes
% ------------------------------------------------------------------------------
The semantics of composed processes is determined by their structure and the involved sub-processes. 

\begin{definition}[Sub-process] 
A sub-process is a process that has been used together with one or many process combinators to build a structurally more complicated process.

\hfill\qedsymbol
\end{definition}

% ------------------------------------------------------------------------------
% choice
% ------------------------------------------------------------------------------
\begin{definition}[Semantics of the choice operator]
\label{def:sem_choice}
Let $Q, R \in \mathcal{P}$ be processes and let them be composed using the choice operator $\vee$. $Q \vee R$ gives a process that behaves like $Q$ and $R$ at the same time and thus the semantics of the resulting process is given by the union of the semantics of $Q$ and $R$.
  \begin{equation}
    \label{eqn:sem_choice}
    sem \left\langle Q \vee R \right\rangle = sem \left\langle Q \right\rangle \cup sem \left\langle R \right\rangle
  \end{equation}
If either $P$ of $Q$ is the error process, the optimistic oracle selects the other process. If both $P$ and $Q$ are the error process $Err$, the oracle must select the error process.
  \begin{eqnarray}
    sem \left\langle P \vee Err \right\rangle & = & sem \left\langle P \right\rangle \\
    sem \left\langle Err \vee Q \right\rangle & = & sem \left\langle Q \right\rangle \\
    sem \left\langle Err \vee Err \right\rangle & = & sem \left\langle Err \right\rangle = \left\{ \bot \right\}
  \end{eqnarray}
  \hfill\qedsymbol
\end{definition}

\begin{example}[Semantics of the choice operator]
\label{exp:sem_chice}
Consider the definitions of $\sigma, \delta, S$ and $D$ from \expref{exp:sem_atomic}. If we compose $S$ and $D$ using the choice operator $\vee$, the semantics of the resulting process is as follows:
  \begin{equation}
    sem \left\langle S \vee D \right\rangle = sem \left\langle S \right\rangle \cup sem \left\langle D \right\rangle = \left\{ \sigma \right\} \cup \left\{ \delta \right\} = \left\{ x \mapsto x+1, x \mapsto 2x \right\}
  \end{equation}
  \hfill\qedsymbol
\end{example}


% ------------------------------------------------------------------------------
% sequence
% ------------------------------------------------------------------------------
\begin{definition}[Semantics of the sequence operator]
\label{def:sem_sequence}
Let $Q, R \in \mathcal{P}$ be processes and let them be composed using the sequence operator, i.e. $R \circ Q$. The semantics of the resulting process is given by function composition. Note that, on the left hand side of \eqnref{eqn:sem_sequence}, $\circ$ stands for the sequence operator for process composition and for function composition on the right hand side.
  \begin{equation}
    \label{eqn:sem_sequence}
    sem \left\langle R \circ Q \right\rangle = \left\{ r \circ q \,|\, q \in sem \left\langle Q \right\rangle, r \in sem \left\langle R \right\rangle \right\}
  \end{equation}
  \hfill\qedsymbol
\end{definition}

\begin{example}[Semantics of the sequence operator]
\label{exp:sem_sequence}
Consider the definitions of $\sigma, \delta, S$ and $D$ from \expref{exp:sem_atomic} and use the sequence operator $\circ$ to compose $D$ and $S$. Since $\circ$ does not have commutative properties, $D \circ S$ and $S \circ D$ will yield different results.
  \begin{equation}
    sem \left\langle D \circ S \right\rangle = \left\{ \delta \circ \sigma \right\} = \left\{ x \mapsto 2x+2 \right\}
  \end{equation}
  \begin{equation}
    sem \left\langle S \circ D \right\rangle = \left\{ \sigma \circ \delta \right\} = \left\{ x \mapsto 2x+1 \right\}
  \end{equation}
  \hfill\qedsymbol
\end{example}


% ------------------------------------------------------------------------------
% parallel
% ------------------------------------------------------------------------------
\begin{definition}[Semantics of the parallel operator]
\label{def:sem_parallel}
Let $Q, R \in \mathcal{P}$ be processes and let them be composed using the parallel operator, i.e. $Q \,|\, R$, with $\rho \left( Q \right) = \rho \left( R \right) = \left( T_a, T_r \right)$. Furthermore, let $\left( T_r, \star \right)$ be a semigroup with the commutative binary operation $\star \colon T_r \times T_r \to T_r$. The semantics of the resulting process is given by running both $Q$ and $R$ on the same argument and combining the elements of their result sets using $\star$.
  \begin{equation}
    \label{eqn:sem_parallel}
    sem \left\langle Q \,|\, R \right\rangle = \left\{ q \star r \,|\, q \in sem \left\langle Q \right\rangle, r \in \left\langle R \right\rangle \right\}.
  \end{equation}
  \hfill\qedsymbol
\end{definition}

\begin{example}[Semantics of the parallel operator]
\label{exp:sem_parallel}
Consider the definitions of $\sigma, \delta, S$ and $D$ from \expref{exp:sem_atomic} and use the parallel operator $|$ to compose $D$ and $S$. The semantics of the result process is determined by the concrete $\star$-operation we employ to combine the processes' results.
  \begin{equation}
    sem \left\langle S \,|\, D \right\rangle = \left\{ \sigma \star \delta \right\}
  \end{equation}
  \hfill\qedsymbol
\end{example}

The reason why we want $\star$ to be commutative is the intuition one might have about parallel composition of processes. For two processes $Q, R \in \mathcal{P}$, it is a reasonable expectation that $Q \,|\, R$ and $R \,|\, Q$ behave exactly the same. Since in the definition of the semantics of parallel composition we employ $\star$ to combine the results of the parallel processes, the properties of $\star$ will be transferred to the parallel composition operator $|$. Hence, if we want $|$ to have commutative properties, we have to require this from $\star$.

Furthermore, for a semigroup $\left(T_r, \star \right)$ that is employed to combine results of processes that are composed using the parallel operator $|$, we require every $t_r \in T_r$, i.e. every concrete value of type $T_r$, to be idempotent respective to $\star$:
\begin{equation}
  \label{eqn:idempotence}
  \bigwedge_{t_r \in T_r} t_r \star t_r = t_r.
\end{equation}

\begin{theorem}
If the property shown in \eqnref{eqn:idempotence} holds, we can guarantee that parallel composition of processes lives up to another reasonable intuition, i.e. parallel composition of a process with itself does not change its semantics. In other words, let $R \in \mathcal{P}$ and compose $R$ using $|$ with itself, then $R \,|\, R$ has the same semantics as $R$.
\end{theorem}

\begin{myproof}
Let $R \in \mathcal{P}$ be a process with $\rho \left( R \right) = \left( T_a, T_r \right)$ and let $\left( T_r, \star \right)$ be a semigroup that satisfies the property given in \eqnref{eqn:idempotence}. Then $sem \left\langle R \,|\, R \right\rangle = sem \left\langle R \right\rangle$.
  \begin{eqnarray*}
    sem \left\langle R \,|\, R \right\rangle & \overset{\defref{def:sem_parallel}}{=} & \left\{ r \star r \,|\, r \in sem \left\langle R \right\rangle \right\} \\
    & \overset{\eqnref{eqn:idempotence}}{=} & \left\{ r \,|\, r \in sem \left\langle R \right\rangle \right\} \\
    & = & sem \left\langle R \right\rangle 
  \end{eqnarray*}
  \hfill$\blacksquare$
\end{myproof}




% ------------------------------------------------------------------------------
% kleene
% ------------------------------------------------------------------------------
Based on the semantics for $\vee$ and $\circ$ from \eqnref{eqn:sem_choice} and \eqnref{eqn:sem_sequence}, the semantics of the \textsc{Kleene} star $*$ is already given, but we will discuss it for the sake completeness. The definition in \chpref{chp:syntax} makes clear the recursive nature of the \textsc{Kleene} star: $R^* = Id \vee R^* \circ R$ and immediately reveals its semantics.
\begin{definition}[Semantics of the \textsc{Kleene} star]
\label{def:sem_kleene}
Let $R \in \mathcal{P}$ be a process and let the \textsc{Kleene} star $*$ be applied to $R$, i.e. $R^*$. The semantics of $R^*$ is then directly derived from \defref{def:static_kleene} and \defref{def:sem_choice}.
  \begin{equation}
    \label{eqn:sem_kleene}
    sem \left\langle R^* \right\rangle = sem \left\langle Id \vee R^* \circ R \right\rangle = sem \left\langle Id \right\rangle \cup sem \left\langle R^* \circ R \right\rangle.
  \end{equation}
  \hfill\qedsymbol
\end{definition}

\begin{example}[Semantics of the \textsc{Kleene} star]
Consider the definition of $\sigma$ and $S$ from \expref{exp:sem_atomic} and apply the \textsc{Kleene} star $*$ to $S$. The semantics of the resulting process $S^*$ includes every number of applications of $\sigma$, or in other words any multiplicity of running $S$.
  \begin{equation}
    sem \left\langle S^* \right\rangle = sem \left\langle Id \vee S^* \circ S \right\rangle = \left\{ x \mapsto x, x \mapsto x+1, x \mapsto x+2, \ldots \right\}
  \end{equation}
  \hfill\qedsymbol
\end{example}

The definition given in \defref{def:sem_kleene} is complete and correct, however it is recursive and contains a non-terminating component. We will give an alternative, yet equivalent definition for the semantics of the \textsc{Kleene} star $*$ that does not have this property. For this purpose, we introduce a shorthand notation for running a process $n$-times.
\begin{definition}[Multiple execution]
\label{def:multiple_execution}
Let $R \in \mathcal{P}$ be a process. Then $R^n$ is the inductively defined $n$-time execution of $R$ with $n \in \mathbb{N}_0$.
  \begin{eqnarray}
    \label{eqn:multiple_execution}
    R^0 & = & Id \\
    R^n & = & R^{n-1} \circ R 
  \end{eqnarray}
  \hfill\qedsymbol
\end{definition}

\begin{definition}[Semantics of the \textsc{Kleene} star (alternative definition)]
\label{def:sem_kleene_alternative}
  Let $R \in \mathcal{P}$ be a process and let the \textsc{Kleene} star $*$ be applied to $R$, i.e. $R^*$. Then, the semantics of $R^*$ is given by the union over the semantics of every finite number of executions of $R$.
  \begin{equation}
    \label{eqn:sem_kleene_alternative}
    sem \left\langle R^* \right\rangle = \bigcup_{i \in \mathbb{N}_0} sem \left\langle R^i \right\rangle
  \end{equation}
  \hfill\qedsymbol
\end{definition}


% ------------------------------------------------------------------------------
% equivalence
% ------------------------------------------------------------------------------
\section{Semantic equivalence}
Based on the definition of process semantics $sem$, we can define a partial ordering $<$ on processes. The partial ordering can then be used to define semantic equivalence on processes.

\begin{definition}[Partial ordering on processes]
\label{def:process_ordering}
Let $P, Q \in \mathcal{P}$ be processes. Then $P$ and $Q$ are in relation $P < Q$ iff the semantics from $P$ can be obtained by removing the undefined value from the semantics of $Q$.
  \begin{equation}
    P < Q \Leftrightarrow sem \left\langle P \right\rangle = sem \left\langle Q \right\rangle \setminus \left\{ \bot \right\}
  \end{equation}
  \hfill\qedsymbol
\end{definition}

If two processes $P$ and $Q$ are in relation $<$, i.e. $P < Q$, this means that $P$ is defined on at least the same values as $Q$ is, $P$'s domain includes $Q$'s domain. Furthermore, if $\bot \in sem \left\langle Q \right\rangle$ then $P$ is defined on values where $Q$ is undefined.

\begin{definition}[Equivalence of processes]
\label{def:process_equivalence}
Let $P, Q \in \mathcal{P}$ be processes. $P$ and $Q$ are equivalent iff their semantics are exactly the same.
  \begin{equation}
    \label{eqn:equivalence}
    P \equiv Q \Leftrightarrow sem \left\langle P \right\rangle = \left\langle Q \right\rangle
  \end{equation}
  \hfill\qedsymbol
\end{definition}

\begin{corollary}[Equivalence of ordered processes]
\label{crl:equivalence_ordered_processes}
Let $P, Q \in \mathcal{P}$ be processes. Then from \defref{def:process_ordering} and \defref{def:process_equivalence} follows that
  \begin{equation}
    P < Q \wedge Q < P \Rightarrow P \equiv Q
  \end{equation}
  and $\bot \notin sem \left\langle P \right\rangle, \bot \notin sem \left\langle Q \right\rangle$. However, from the equivalence of two processes $P, Q$ it does not automatically follow that they are ordered since the semantics of $P$ and $Q$ can contain $\bot$.
  \begin{equation}
    P \equiv Q \not\Rightarrow P < Q \wedge Q < P
\end{equation}    
  \hfill\qedsymbol
\end{corollary}

\begin{definition}[Process substitution]
\label{def:process_substitution}
Let $P, Q, R \in \mathcal{P}$ be processes and let $Q$ be a sub-process of $P$, i.e. $P$ contains $Q$ at some point in its structure. Then we can transform $P$ into a new process by replacing every occurrence of $Q$ with $R$ and write $P_{\left[ Q / R \right]}$ for this.

\hfill\qedsymbol
\end{definition}

\begin{corollary}[Substitution with an equivalent process]
\label{crl:process_substitution}
Let $P, Q, R \in \mathcal{P}$ be processes and $Q \equiv R$. Then it follows from \defref{def:process_equivalence} and \defref{def:process_substitution} that $P_{\left[ Q / R \right]}$ is equivalent to $P$ iff $Q$ and $R$ are equivalent:
  \begin{equation}
    P \equiv P_{\left[ Q / R \right]} \Leftrightarrow Q \equiv R.
  \end{equation}
  \hfill\qedsymbol
\end{corollary}

\begin{corollary}[Substitution of ordered processes]
Let $P, Q, R, S \in \mathcal{P}$ be processes and $R < Q$. Then from \crlref{crl:equivalence_ordered_processes} and \defref{def:process_substitution} it follows that
  \begin{equation}
    R < Q \Rightarrow P_{\left[ Q / R \right]} \equiv P \lor P_{\left[ Q / R \right]} < P
  \end{equation}
  depending on $P$'s sub-processes and their semantics. If $P$ involves a sub-process $S$ that is not a sub-process of $Q$ and $\bot \in sem \left\langle S \right\rangle$, then $P$ can still produce $\bot$ and it follows $P \equiv P_{\left[Q / R \right]}$. This is the case because of the strictness of processes in $\bot$, as defined in \defref{def:process_strictness}. If there is no such sub-process involved in $P$, then $P_{\left[Q / R \right]} < P$.
    
  \hfill\qedsymbol
\end{corollary}

The equivalence from \crlref{crl:process_substitution} can be used to perform transformations on processes and optimise them according to some measurable property. E.g. assume that we have a process $P$ that involved $Q$ as a sub-process. Now if $Q$ has high time complexity and we have another process $R$ with low time complexity and $Q \equiv R$, then we might want to use $P_{\left[ Q / R \right]}$ in a productive system instead of $P$ in order to save run time.

\begin{definition}[Inverse process]
\label{def:inverse_process}
Let $P \in \mathcal{P}$ be a process. Then an inverse process $\overline{P}$ for $P$ is a process that satisfies the following property:
  \begin{equation}
    \overline{P} \circ P \equiv Id.
  \end{equation}
  \hfill\qedsymbol
\end{definition}

As a final remark, we want to point out that, for a process $P$ an inverse process as defined in \defref{def:inverse_process} does generally \textbf{not} exist since not every computable function is invertible\footnote{Take, e.g. the function $f \colon x \mapsto x^2$. Clearly, $f$ is not injective since $f \left( x \right) = f \left( -x \right)$ and therefore not invertible. However, there is an inverse relation $\overline{f_r} \colon x \mapsto \left\{ \overline{x} \,|\, f \left( \overline{x} \right) = x \right\}$ that maps a value $x$ to the set of values that, if $f$ is applied to them, will yield $x$.}.





% ==============================================================================
% laws
% ==============================================================================

\section{Laws}
\label{chp:laws}
Based on the defined semantics of our process algebra, i.e. $sem$, we can find a set of laws that allow us to transform processes into an equivalent representation. This is useful for two cases: on the one hand, it gives the developer who is using our algebra a tool to reason about his programs and transform them into equivalent representations. On the other hand, it is possible to implement an automated process optimiser that can, based on measurable properties, transform processes into more efficient representations while maintaining their semantics.

In the following, let $P, Q, R \in \mathcal{P}$ and assume that their types are suitable for the desired composition in the respective case. Proofs for the stated laws can be found in \appref{chp:proofs}.

\subsection{Associativity}
The associativity property holds for the choice operator $\vee$, the parallel operator $|$ and the sequence operator $\circ$ the the \textsc{Hive} process algebra.
\begin{equation*}
\begin{array}{rcccl}
  \left( R \vee Q \right) \vee P & \equiv & R \vee Q \vee P & \equiv & R \vee \left( Q \vee P \right) \\
  \left( R \,|\, Q \right) |\, P & \equiv & R \,|\, Q \,|\, P & \equiv & R \,| \left( Q \,|\, P \right) \\
  \left( R \circ Q \right) \circ P & \equiv & R \circ Q \circ P & \equiv & R \circ \left( Q \circ P \right)
\end{array}
\end{equation*}

\subsection{Distributivity}
In our algebra, the choice operator $\vee$, the parallel operator $|$ and the sequence operator $\circ$ distribute over the choice operator $\vee$. Furthermore, the sequence operator $\circ$ distributes over the parallel operator $|$.
\begin{eqnarray*}
  P \vee \left( Q \vee R \right) & \equiv & \left( P \vee Q \right) \vee \left( P \vee R \right) \\
  P \,| \left( Q \vee R \right) & \equiv & \left( P \,|\, Q \right) \vee \left( P \,|\, R \right) \\
  P \circ \left( Q \vee R \right) & \equiv & \left( P \circ Q \right) \vee \left( P \circ R \right) \\
  & & \\
  P \circ \left( Q \,|\, R \right) & \equiv & \left( P \circ Q \right) | \left( P \circ R \right)
\end{eqnarray*}

\subsection{Commutativity}
The commutativity property is given for the choice operator $\vee$ and the parallel operator $|$, but not for the sequence operator $\circ$.
\begin{eqnarray*}
  P \vee Q & \equiv & Q \vee P \\
  P \,|\, Q & \equiv & Q \,|\, P
\end{eqnarray*}

\subsection{Idempotence}
Idempotence properties can be dependant on both the operation at hand and a specific element. For every process $P$, if it is combined with itself using the choice operator $\vee$, the resulting process has the same semantics as the original process $P$. The identity process $Id$ as well as the error process $Err$ is idempotent regarding sequential composition $\circ$ with itself. Furthermore, multiple application of the \textsc{Kleene} star $*$ does not change the semantics of a process compared to single application of $*$.
\begin{eqnarray*}
  P \vee P & \equiv & P \\
  Id \circ Id & \equiv & Id \\
  Err \circ Err & \equiv & Err \\
  (P^*)^* & \equiv & P^*
\end{eqnarray*}

\subsection{Neutral elements}
The identity process $Id$ is both a left and right neutral element for sequential composition $\circ$. Furthermore, it follows directly from the definition of the semantics of the error process $Err$ and the choice operator $\vee$ that $Err$ is a neutral element for $\vee$. The commutative property of $\vee$ makes $Err$ both a left and right neutral element.
\begin{eqnarray*}
  Id \circ P & \equiv & P \\
  P \circ Id & \equiv & P
\end{eqnarray*}