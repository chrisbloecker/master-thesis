\chapter{Conclusion}
\vspace*{-0.5em}
Our goal was the design of a process calculus that allows the description of parallel programs on a high level of abstraction while hiding technical details about parallelisation and process communication, which we have reached. We have taken a look at two prominent process calculi, namely \textsc{CCS} and \textsc{CSP}, but found that they are unsuitable for our purposes. They both involve a notation of process communication and non-determinism, which is an undesired property for application in practice.

\vspace*{-0.25em}
We have designed our own process calculus and used \textsf{Haskell} for a direct implementation. The process calculus introduces a small set of process combinators that are used to compose processes. Each of the process combinators involves rules concerning its static semantics. These rules have been represented in \textsf{Haskell} using a generalised algebraic data type\index{Generalised algebraic data type} that only allows composition of suitable processes. The semantics of processes has been defined inductively, based on the used process combinator and the semantics of the sub-processes. For the implementation of process semantics in \textsf{Haskell}, given by an interpreter, we have chosen two variants: \textsf{Concurrent Haskell} and \textsf{Cloud Haskell}. The implementation has been straight forward, directly reflecting the formal definition. We found that \textsf{Haskell} was particularly well suited for this.

\vspace*{-0.25em}
To illustrate the usability of our process calculus, we have given two examples. One of them being a parallel interpreter for arithmetic expressions and the other one a prototype of an artificial ant system. We have created three variants of the ant system: a parallelised variant using our implementation based on \textsf{Concurrent Haskell}, a distributed variant using our implementation based on \textsf{Cloud Haskell} and a sequential variant.

\vspace*{-0.25em}
Our tests have shown that the ant system using our implementation based on \textsf{Concurrent Haskell} achieves a speedup compared to the sequential version. The ant system using our implementation based on \textsf{Cloud Haskell} has not achieved a speedup but taken significantly longer than the sequential variant. 

\vspace*{-0.25em}
We conclude that our process calculus is mature enough for practical application and to lead to a speedup when used for parallelisation. However, our implementation based on \textsf{Cloud Haskell} introduces too much overhead in the conducted tests and couldn't lead to a speedup, we're afraid this might be an indication for practical inapplicability. For future work, it remains to be seen if this is the case in general or only in our test case.
\vspace*{-0.5em}