\chapter{Proofs}
\label{chp:proofs}

\vfill
\begin{theorem}[Associative property of sequence composition]
\label{thm:associativity_sequence}
Sequence composition has associative property, i.e. for any three processes $P, Q, R \in \mathcal{P}$ with suitable type signatures, the following equivalence holds:
\begin{equation*}
  \sequence{P}{\sequence{Q}{R}} \equiv \sequence{\sequence{P}{Q}}{R}.
\end{equation*}
\end{theorem}

\begin{myproof}[Associative property of sequence composition]
Let $P, Q, R \in \mathcal{P}$ be processes with type signatures $\typesignature{P}{a}{b}, \typesignature{Q}{b}{c}, \typesignature{R}{c}{d}$.

Then $\sem{\sequence{P}{\sequence{Q}{R}}} = \sem{\sequence{\sequence{P}{Q}}{R}}$.
\begin{eqnarray*}
  & & \sem{\sequence{P}{\sequence{Q}{R}}} \\
  & \overset{\defref{def:sem_sequence}}{=} & \sem{\sequence{Q}{R}} \circ \sem{P} \\
  & \overset{\defref{def:sem_sequence}}{=} & \left( \sem{R} \circ \sem{Q} \right) \circ \sem{P} \\
  & \overset{\text{associativity of }\circ}{=} & \sem{R} \circ \left( \sem{Q} \circ \sem{P} \right) \\
  & \overset{\defref{def:sem_sequence}}{=} & \sem{R} \circ \sem{\sequence{P}{Q}} \\
  & \overset{\defref{def:sem_sequence}}{=} & \sem{\sequence{\sequence{P}{Q}}{R}}
\end{eqnarray*}

\hfill$\blacksquare$
\end{myproof}


\clearpage
\begin{theorem}[Distributivity of parallel composition over choice composition]
\label{thm:distributivity_parallel_choice}
Parallel composition distributes over choice composition, i.e. for any processes $B \in \mathcal{B}$ and $C, P, Q, R \in \mathcal{P}$ with suitable type signatures, the following equivalence holds:
\begin{equation*}
  \parallel{C}{P}{\choice{B}{Q}{R}} \equiv \choice{B}{\parallel{C}{P}{Q}}{\parallel{C}{P}{R}}.
\end{equation*}
\end{theorem}

\begin{myproof}[Distributivity of parallel composition over choice composition]
Let $B \in \mathcal{B}$ and $C, P, Q, R \in \mathcal{P}$ be processes with type signatures $\typesignature{B}{a}{T_{Boolean}}, \typesignature{C}{\left( b, c \right)}{d}, \typesignature{P}{a}{b}, \typesignature{Q}{a}{c}, \typesignature{R}{a}{c}$.

Then $\sem{\parallel{C}{P}{\choice{B}{Q}{R}}} = \sem{\choice{B}{\parallel{C}{P}{Q}}{\parallel{C}{P}{R}}}$.
\begin{eqnarray*}
  &                                        & \sem{\parallel{C}{P}{\choice{B}{Q}{R}}} \\
  & \overset{\defref{def:sem_parallel}}{=} & x \mapsto \sem{C} \left( \sem{P} \left( x \right), \sem{\choice{B}{Q}{R}} \left( x \right) \right) \\
  & \overset{\defref{def:sem_choice}}{=}   & x \mapsto \sem{C} \left( \sem{P} \left( x \right), x \mapsto \left\{\begin{array}{ll}
                                                                                 \sem{Q} \left( x \right) & \text{if } B \left( x \right) \\
                                                                                 \sem{R} \left( x \right) & \text{if } \overline{B \left( x \right)} \\
                                                                                 \bot & \text{otherwise}
                                                                               \end{array}
                                                                        \right.\right) \\
  & =                                      & x \mapsto \left\{\begin{array}{ll}
                                                      \sem{C} \left( \sem{P} \left( x \right), \sem{Q} \left( x \right) \right) & \text{if } B \left( x \right) \\
                                                      \sem{C} \left( \sem{P} \left( x \right), \sem{R} \left( x \right) \right) & \text{if } \overline{B \left( x \right)} \\
                                                      \bot & \text{otherwise}
                                                    \end{array}
                                             \right. \\
  & \overset{\defref{def:sem_parallel}}{=} & x \mapsto \left\{\begin{array}{ll}
                                                      \sem{\parallel{C}{P}{Q}} \left( x \right) & \text{if } B \left( x \right) \\
                                                      \sem{\parallel{C}{P}{R}} \left( x \right) & \text{if } \overline{B \left( x \right)} \\
                                                      \bot & \text{otherwise}
                                                    \end{array}
                                             \right. \\
  & \overset{\defref{def:sem_choice}}{=}   & \sem{\choice{B}{\parallel{C}{P}{Q}}{\parallel{C}{P}{R}}}
\end{eqnarray*}

\hfill$\blacksquare$
\end{myproof}

\begin{theorem}[Distributivity of sequence composition over choice composition]
\label{thm:distributivity_sequence_choice}
Sequence composition distributes over choice composition, i.e. for any processes $B \in \mathcal{B}$ and $P, Q, R \in \mathcal{P}$ with suitable type signatures, the following equivalence holds:
\begin{equation*}
  \sequence{P}{\choice{B}{Q}{R}} \equiv \choice{B}{\sequence{P}{Q}}{\sequence{P}{R}}.
\end{equation*}
\end{theorem}

\begin{myproof}[Distributivity of sequence composition over choice composition]
Let $B \in \mathcal{B}$ and $P, Q, R \in \mathcal{P}$ be processes with type signatures $\typesignature{B}{a}{T_{Boolean}}, \typesignature{P}{a}{b}, \typesignature{Q}{b}{c}, \typesignature{R}{b}{c}$. 

Then $\sem{\sequence{P}{\choice{B}{Q}{R}}} = \sem{\choice{B}{\sequence{P}{Q}}{\sequence{P}{R}}}$.
\begin{eqnarray*}
  & & \sem{\sequence{P}{\choice{B}{Q}{R}}} \\
  & \overset{\defref{def:sem_sequence}}{=} & \sem{\choice{B}{Q}{R}} \circ \sem{P} \\
  & \overset{\defref{def:sem_choice}}{=}   & \left( x \mapsto \left\{\begin{array}{ll}
                                                                       \sem{Q} \left( x \right) & \text{if } B \left( x \right) \\
                                                                       \sem{R} \left( x \right) & \text{if } \overline{B \left( x \right)} \\
                                                                       \bot & \text{otherwise}
                                                                     \end{array}
                                                              \right.\right) \circ \sem{P} \\
  & = & \left\{\begin{array}{ll}
                 \left( x \mapsto \sem{Q} \left( x \right) \right) \circ \sem{P} & \text{if } B \left( x \right) \\
                 \left( x \mapsto \sem{R} \left( x \right) \right) \circ \sem{P} & \text{if } \overline{B \left( x \right)} \\
                 \left( x \mapsto \bot \right) \circ \sem{P} & \text{otherwise}
               \end{array}
        \right. \\
  & \overset{\defref{def:sem_sequence}, ToDo}{=} & x \mapsto \left\{\begin{array}{ll}
                                                                      \sem{\sequence{P}{Q}} \left( x \right) & \text{if } B \left( x \right) \\
                                                                      \sem{\sequence{P}{R}} \left( x \right) & \text{if } \overline{B \left( x \right)} \\
                                                                      \bot & \text{otherwise}
                                                          \end{array}
                                                   \right. \\
  & \overset{\defref{def:sem_choice}}{=}   & \sem{\choice{B}{\sequence{P}{Q}}{\sequence{P}{R}}}
\end{eqnarray*}


\hfill$\blacksquare$
\end{myproof}

\begin{theorem}[Distributivity of sequence composition over parallel composition]
\label{thm:distributivity_sequence_parallel}
Sequence composition distributes over parallel composition, i.e. for any processes $C, P, Q, R \in \mathcal{P}$ with suitable type signatures, the following equivalence holds:
\begin{equation*}
  \sequence{P}{\parallel{C}{Q}{R}} \equiv \parallel{C}{\sequence{P}{Q}}{\sequence{P}{R}}.
\end{equation*}
\end{theorem}

\begin{myproof}[Distributivity of sequence composition over parallel composition]
Let $C, P, Q, R \in \mathcal{P}$ be processes with type signatures $\typesignature{C}{\left( c, d \right)}{e}, \typesignature{P}{a}{b}, \typesignature{Q}{b}{c}, \typesignature{R}{b}{d}$.

Then $\sem{\sequence{P}{\parallel{C}{Q}{R}}}{x} = \sem{\parallel{C}{\sequence{P}{Q}}{\sequence{P}{R}}}{x}$.
\begin{eqnarray*}
  & & \sem{\sequence{P}{\parallel{C}{Q}{R}}}{x} \\
  & \overset{\defref{def:sem_sequence}}{=} & \sem{\parallel{C}{Q}{R}}{\sem{P}{x}} \\
  & \overset{\defref{def:sem_parallel}}{=} & \sem{C}{\left( \sem{Q}{\sem{P}{x}}, \sem{R}{\sem{P}{x}} \right)} \\
  & \overset{\defref{def:sem_sequence}}{=} & \sem{C}{\left( \sem{\sequence{P}{Q}}{x}, \sem{\sequence{P}{R}}{x} \right)} \\
  & \overset{\defref{def:sem_parallel}}{=} & \sem{\parallel{C}{\sequence{P}{Q}}{\sequence{P}{R}}}{x}
\end{eqnarray*}

\hfill$\blacksquare$
\end{myproof}


\begin{theorem}[Idempotence of choice composition]
\label{thm:idempotence_choice}
For any predicate $B \in \mathcal{B}$ and any process $P \in \mathcal{P}$, choice composition of $P$ with itself yields a process that behaves like $P$, regardless of the used predicate $B$.
\begin{equation*}
  \choice{B}{P}{P} \equiv P
\end{equation*}
\end{theorem}

\begin{myproof}[Idempotence of choice composition]
Let $B \in \mathcal{B}$ and $P \in \mathcal{P}$ be processes with type signatures $\typesignature{B}{a}{T_{Boolean}}, \typesignature{P}{a}{b}$. 

Then $\sem{\choice{B}{P}{P}}{x} = \sem{P}{x}$.
\begin{eqnarray*}
  & & \sem{\choice{B}{P}{P}}{x} \\
    & \overset{\defref{def:sem_choice}}{=} & \left\{\begin{array}{ll}
                                                      \sem{P}{x} & \text{if } B \left( x \right) \\
                                                      \sem{P}{x} & \text{if } \overline{B \left( x \right)}
                                                    \end{array}
                                             \right. \\
    & = & \sem{P}{x}
\end{eqnarray*}

\hfill$\blacksquare$
\end{myproof}

%\begin{theorem}[Idempotence of sequence composition respecting the identity process]
%\label{thm:idempotence_identity}
%Sequential composition of the identity process $Id$ with itself yields the identity process.
%\begin{equation*}
%  \sequence{Id}{Id} \equiv Id.
%\end{equation*}
%\end{theorem}

%\begin{myproof}[Idempotence of sequence composition respecting the identity process]
%Let $Id \in \mathcal{P}$ be the identity process and compose it with itself using sequence composition. The resulting process is the identity process $Id$, i.e. $\sem{\sequence{Id}{Id}}{x} = \sem{Id}{x}$.
%\begin{eqnarray*}
%  & & \sem{\sequence{Id}{Id}}{x} \\
%    & \overset{\defref{def:sem_sequence}}{=} & \sem{Id}{\sem{Id}{x}} \\
%    & \overset{\defref{def:sem_id}}{=} & \sem{Id}{x}
%\end{eqnarray*}
%
%\hfill$\blacksquare$
%\end{myproof}


\begin{theorem}[Idempotence of sequence composition respecting the error process]
\label{thm:idempotence_identity}
Sequential composition of the error process $Err$ with itself yields the error process.
\begin{equation*}
  \sequence{Err}{Err} \equiv Err.
\end{equation*}
\end{theorem}

\begin{myproof}[Idempotence of sequence composition respecting the identity process]
Let $Id \in \mathcal{P}$ be the identity process and compose it with itself using sequence composition. The resulting process is the identity process $Id$, i.e. $\sem{\sequence{Id}{Id}}{x} = \sem{Id}{x}$.
\begin{eqnarray*}
  a
\end{eqnarray*}

\hfill$\blacksquare$
\end{myproof}


\begin{theorem}[$Id$ is a left and right unit for sequence composition]
\label{thm:idempotence_identity}
Sequential composition of any process $P \in \mathcal{P}$ with the identity process $Id$ on the left or on the right yields $P$.
\begin{equation*}
  \sequence{Id}{P} \equiv P \equiv \sequence{P}{Id}
\end{equation*}
\end{theorem}

\begin{myproof}[$Id$ is a left and right unit for sequence composition]
Let $P \in \mathcal{P}$ be a process and $Id$ be the identity process. $Id$ is a left and right unit for sequence composition.
\begin{eqnarray*}
  & & \sem{\sequence{Id}{P}}{x} \\
    & \overset{\defref{def:sem_sequence}}{=} & \sem{P}{\sem{Id}{x}} \\
    & \overset{\defref{def:sem_id}}{=} & \sem{P}{x} \\
    & \overset{\defref{def:sem_id}}{=} & \sem{Id}{\sem{P}{x}} \\
    & \overset{\defref{def:sem_sequence}}{=} & \sem{\sequence{P}{Id}}{x} \\
\end{eqnarray*}

\hfill$\blacksquare$
\end{myproof}

\begin{corollary}[Idempotence of sequence composition respecting the identity process]
Since $Id$ is both a left and right unit of sequence composition, $Id$ is an idempotent element respecting sequence composition, i.e. for any process $P \in \mathcal{P}$, the following equation holds:
\begin{equation*}
  \sequence{Id}{P} \equiv \sequence{P}{Id}.
\end{equation*}

\hfill$\blacksquare$
\end{corollary}