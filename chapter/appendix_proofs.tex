\chapter{Proofs}
\label{chp:proofs}
The proofs in this appendix establish the laws proposed in \chpref{chp:laws} as well as the theorem on process equivalence after substitution of a sub-process with an equivalent sub-process from \chpref{chp:semantic_equivalence}.

\vfill
\begin{theorem}[Associative property of sequential composition]
\label{thm:associativity_sequence}
Sequential composition has associative property, i.e. for any three processes $P, Q, R \in \mathcal{P}$ with suitable type signatures, the following equivalence holds:
\begin{equation*}
  \sequence{P}{\sequence{Q}{R}} \equiv \sequence{\sequence{P}{Q}}{R}.
\end{equation*}
\hfill$\qedsymbol$
\end{theorem}

\begin{myproof}[Associative property of sequential composition]
Let $P, Q, R \in \mathcal{P}$ be processes with type signatures $\typesignature{P}{a}{b}, \typesignature{Q}{b}{c}, \typesignature{R}{c}{d}$.

Then $\sem{\sequence{P}{\sequence{Q}{R}}} = \sem{\sequence{\sequence{P}{Q}}{R}}$.
\begin{eqnarray*}
  & & \sem{\sequence{P}{\sequence{Q}{R}}} \\
  & \overset{\defref{def:sem_sequence}}{=} & \sem{\sequence{Q}{R}} \circ \sem{P} \\
  & \overset{\defref{def:sem_sequence}}{=} & \left( \sem{R} \circ \sem{Q} \right) \circ \sem{P} \\
  & \overset{\text{associativity of }\circ}{=} & \sem{R} \circ \left( \sem{Q} \circ \sem{P} \right) \\
  & \overset{\defref{def:sem_sequence}}{=} & \sem{R} \circ \sem{\sequence{P}{Q}} \\
  & \overset{\defref{def:sem_sequence}}{=} & \sem{\sequence{\sequence{P}{Q}}{R}}
\end{eqnarray*}
Thus, \thmref{thm:associativity_sequence} is valid. \hfill$\blacksquare$
\end{myproof}


\clearpage
\begin{theorem}[Distributivity of parallel composition over choice composition]
\label{thm:distributivity_parallel_choice}
Parallel composition distributes over choice composition, i.e. for any predicate $B$ and processes $C, P, Q, R \in \mathcal{P}$ with suitable type signatures, the following equivalence holds:
\begin{equation*}
  \parallel{C}{P}{\choice{B}{Q}{R}} \equiv \choice{B}{\parallel{C}{P}{Q}}{\parallel{C}{P}{R}}.
\end{equation*}

\vspace*{-1em}
\hfill$\qedsymbol$
\end{theorem}

\begin{myproof}[Distributivity of parallel composition over choice composition]
Let $B$ be a predicate and let $C, P, Q, R \in \mathcal{P}$ be processes with type signatures $\typesignature{B}{a}{T_{Boolean}}, \linebreak \typesignature{C}{\left( b, c \right)}{d}, \typesignature{P}{a}{b}, \typesignature{Q}{a}{c}, \typesignature{R}{a}{c}$.

Then $\sem{\parallel{C}{P}{\choice{B}{Q}{R}}} = \sem{\choice{B}{\parallel{C}{P}{Q}}{\parallel{C}{P}{R}}}$.
\begin{eqnarray*}
  &                                        & \sem{\parallel{C}{P}{\choice{B}{Q}{R}}} \\
  & \overset{\defref{def:sem_parallel}}{=} & x \mapsto \sem{C} \left( \sem{P} \left( x \right), \sem{\choice{B}{Q}{R}} \left( x \right) \right) \\
  & \overset{\defref{def:sem_choice}}{=}   & x \mapsto \sem{C} \left( \sem{P} \left( x \right), x \mapsto \left\{\begin{array}{ll}
                                                                                 \sem{Q} \left( x \right) & \text{if } B \left( x \right) \\
                                                                                 \sem{R} \left( x \right) & \text{if } \overline{B \left( x \right)} \\
                                                                                 \bot & \text{otherwise}
                                                                               \end{array}
                                                                        \right.\right) \\
  & =                                      & x \mapsto \left\{\begin{array}{ll}
                                                      \sem{C} \left( \sem{P} \left( x \right), \sem{Q} \left( x \right) \right) & \text{if } B \left( x \right) \\
                                                      \sem{C} \left( \sem{P} \left( x \right), \sem{R} \left( x \right) \right) & \text{if } \overline{B \left( x \right)} \\
                                                      \bot & \text{otherwise}
                                                    \end{array}
                                             \right. \\
  & \overset{\defref{def:sem_parallel}}{=} & x \mapsto \left\{\begin{array}{ll}
                                                      \sem{\parallel{C}{P}{Q}} \left( x \right) & \text{if } B \left( x \right) \\
                                                      \sem{\parallel{C}{P}{R}} \left( x \right) & \text{if } \overline{B \left( x \right)} \\
                                                      \bot & \text{otherwise}
                                                    \end{array}
                                             \right. \\
  & \overset{\defref{def:sem_choice}}{=}   & \sem{\choice{B}{\parallel{C}{P}{Q}}{\parallel{C}{P}{R}}}
\end{eqnarray*}

\vspace*{-1em}
Thus, \thmref{thm:distributivity_parallel_choice} is valid. \hfill$\blacksquare$
\end{myproof}


\begin{theorem}[Distributivity of sequential composition over parallel composition]
\label{thm:distributivity_sequence_parallel}
Sequential composition distributes over parallel composition, i.e. for any processes $C, P, Q, R \in \mathcal{P}$ with suitable type signatures, the following equivalence holds:
\begin{equation*}
  \sequence{P}{\parallel{C}{Q}{R}} \equiv \parallel{C}{\sequence{P}{Q}}{\sequence{P}{R}}.
\end{equation*}

\vspace*{-1em}
\hfill$\qedsymbol$
\end{theorem}

\begin{myproof}[Distributivity of sequential composition over parallel composition]
Let $C, P, Q, R \in \mathcal{P}$ be processes with type signatures $\typesignature{C}{\left( c, d \right)}{e}, \typesignature{P}{a}{b}, \linebreak \typesignature{Q}{b}{c}, \typesignature{R}{b}{d}$.

Then $\sem{\sequence{P}{\parallel{C}{Q}{R}}} = \sem{\parallel{C}{\sequence{P}{Q}}{\sequence{P}{R}}}$.
\begin{eqnarray*}
  & & \sem{\sequence{P}{\parallel{C}{Q}{R}}} \\
  & \overset{\defref{def:sem_sequence}}{=}   & \sem{\parallel{C}{Q}{R}} \circ \sem{P} \\
  & \overset{\defref{def:sem_parallel}}{=}   & x \mapsto \sem{C} \left( \sem{Q} \left( x \right), \sem{R} \left( x \right) \right) \circ \sem{P} \\
  & \overset{\text{function composition}}{=} & x \mapsto \sem{C} \left( \sem{Q} \left( \sem{P} \left( x \right) \right), \sem{R} \left( \sem{P} \left( x \right) \right) \right) \\
  & \overset{\defref{def:sem_sequence}}{=}   & x \mapsto \sem{C} \left( \sem{\sequence{P}{Q}} \left( x \right), \sem{\sequence{P}{R}} \left( x \right) \right) \\
  & \overset{\defref{def:sem_parallel}}{=}   & \sem{\parallel{C}{\sequence{P}{Q}}{\sequence{P}{R}}}
\end{eqnarray*}

\vspace*{-1em}
Thus, \thmref{thm:distributivity_sequence_parallel} is valid. \hfill$\blacksquare$
\end{myproof}


\begin{theorem}[Distributivity of sequential composition over choice composition]
\label{thm:distributivity_sequence_choice}
\text{} \linebreak Sequential composition distributes over choice composition, i.e. for any predicate $B$ and processes $P, Q, R \in \mathcal{P}$ with suitable type signatures, the following equivalence holds:
\begin{equation*}
  \sequence{P}{\choice{B}{Q}{R}} \equiv \choice{B}{\sequence{P}{Q}}{\sequence{P}{R}}.
\end{equation*}

\vspace*{-0.75em}
\hfill$\qedsymbol$
\end{theorem}

\begin{myproof}[Distributivity of sequential composition over choice composition]
Let $B$ be a predicate and let $P, Q, R \in \mathcal{P}$ be processes with type signatures $\typesignature{B}{a}{T_{Boolean}}, \linebreak \typesignature{P}{a}{b}, \typesignature{Q}{b}{c}, \typesignature{R}{b}{c}$. 

Then $\sem{\sequence{P}{\choice{B}{Q}{R}}} = \sem{\choice{B}{\sequence{P}{Q}}{\sequence{P}{R}}}$.
\begin{eqnarray*}
  & & \sem{\sequence{P}{\choice{B}{Q}{R}}} \\
  & \overset{\defref{def:sem_sequence}}{=} & \sem{\choice{B}{Q}{R}} \circ \sem{P} \\
  & \overset{\defref{def:sem_choice}}{=}   & \left( x \mapsto \left\{\begin{array}{ll}
                                                                       \sem{Q} \left( x \right) & \text{if } B \left( x \right) \\
                                                                       \sem{R} \left( x \right) & \text{if } \overline{B \left( x \right)} \\
                                                                       \bot & \text{otherwise}
                                                                     \end{array}
                                                              \right.\right) \circ \sem{P} \\
  & = & \left\{\begin{array}{ll}
                 \left( x \mapsto \sem{Q} \left( x \right) \right) \circ \sem{P} & \text{if } B \left( x \right) \\
                 \left( x \mapsto \sem{R} \left( x \right) \right) \circ \sem{P} & \text{if } \overline{B \left( x \right)} \\
                 \left( x \mapsto \bot \right) \circ \sem{P} & \text{otherwise}
               \end{array}
        \right. \\
  & \overset{\defref{def:sem_sequence}}{=} & x \mapsto \left\{\begin{array}{ll}
                                                                      \sem{\sequence{P}{Q}} \left( x \right) & \text{if } B \left( x \right) \\
                                                                      \sem{\sequence{P}{R}} \left( x \right) & \text{if } \overline{B \left( x \right)} \\
                                                                      \bot & \text{otherwise}
                                                          \end{array}
                                                   \right. \\
  & \overset{\defref{def:sem_choice}}{=}   & \sem{\choice{B}{\sequence{P}{Q}}{\sequence{P}{R}}}
\end{eqnarray*}

\vspace*{-0.75em}
Thus, \thmref{thm:distributivity_sequence_choice} is valid. \hfill$\blacksquare$
\end{myproof}


\begin{theorem}[$Id$ is a left and right neutral element of sequential composition]
\label{thm:idempotence_identity}
Sequential composition of any process $P \in \mathcal{P}$ with the identity process $Id$ on the left or on the right yields $P$.
\begin{equation*}
  \sequence{Id}{P} \equiv P \equiv \sequence{P}{Id}
\end{equation*}

\vspace*{-0.75em}
\hfill$\qedsymbol$
\end{theorem}

\begin{myproof}[$Id$ is a left and right neutral element of sequential composition]
\label{prf:neutral_id}
Let $P \in \mathcal{P}$ be a process and $Id$ be the identity process. $Id$ is a left and right neutral element of sequential composition.
\begin{eqnarray*}
  & & \sem{\sequence{Id}{P}} \\
    & \overset{\defref{def:sem_sequence}}{=} & \sem{P} \circ \sem{Id} \\
    & \overset{\defref{def:sem_id}}{=} & \sem{P} \circ \left( x \mapsto x \right) \\
    & \overset{\text{composition with identity function}}{=} & \sem{P} \\
    & \overset{\text{composition with identity function}}{=} & \left( x \mapsto x \right) \circ \sem{P} \\
    & \overset{\defref{def:sem_id}}{=} & \sem{Id} \circ \sem{P} \\
    & \overset{\defref{def:sem_sequence}}{=} & \sem{\sequence{P}{Id}} \\
\end{eqnarray*}

\vspace*{-2em}
Thus, \thmref{thm:idempotence_identity} is valid. \hfill$\blacksquare$
\end{myproof}

\clearpage

\begin{corollary}[Idempotence of the identity process with respect to sequential composition]
Since $Id$ is both a left and right neutral element of sequential composition, $Id$ is an idempotent element respecting sequential composition. The following equation follows directly from proof \ref{prf:neutral_id}:
\begin{equation*}
  \sem{\sequence{Id}{Id}} = \sem{Id}.
\end{equation*}

\hfill$\blacksquare$
\end{corollary}


\begin{theorem}[Idempotence of the error process with respect to sequential composition]
\label{thm:idempotence_err}
The error process $Err$ is an idempotent element of sequential composition, i.e. sequential composition of the error process with itself yields the error process.
\begin{equation*}
  \sequence{Err}{Err} \equiv Err.
\end{equation*}

\hfill$\qedsymbol$
\end{theorem}

\begin{myproof}[Idempotence of the error process with respect to sequential composition]
Let $Err$ be the error process. Sequential composition of $Err$ with itself yields $Err$, i.e. $\sem{\sequence{Err}{Err}} = \sem{Err}$.
\begin{eqnarray*}
  & & \sem{\sequence{Err}{Err}} \\
    & \overset{\defref{def:sem_sequence}}{=} & \sem{Err} \circ \sem{Err} \\
    & \overset{\defref{def:sem_err}}{=} & \left( x \mapsto \bot \right) \circ \left( x \mapsto \bot \right) \\
    & \overset{\text{function composition}}{=} & x \mapsto \bot \\
    & \overset{\defref{def:sem_err}}{=} & \sem{Err}
\end{eqnarray*}

Thus, \thmref{thm:idempotence_err} is valid. \hfill$\blacksquare$
\end{myproof}



\begin{theorem}[Substitution with an equivalent process leaves semantics unchanged]
\label{thm:substitution_equivalence}
Let $P, Q, R \in \mathcal{P}$ be processes and let $Q \equiv R$. Then, substituting one or more occurrences of $Q$ in $P$ with $R$ leaves $\sem{P}$ unchanged:
  \begin{equation*}
    Q \equiv R \rightarrow P \equiv P_{\left[ Q / R \right]}.
  \end{equation*}
  
  \hfill$\qedsymbol$
\end{theorem}

\begin{myproof}[Substitution with an equivalent process leaves semantics unchanged]
\label{prf:substitution}
In the following, let $P, Q, R, S, T \in \mathcal{P}$ be processes with $Q \equiv R$.

Consider the case that $Q$ is not a sub-process of $P$. Then, trivially $P \equiv P_{\left[ Q / R \right]}$ holds.

Now consider the case that $Q$ is a sub-process of $P$. The claimed property can be shown by structural induction:
\begin{enumerate}
  \item $P = Q$ is a basic process with intrinsic function $f_Q$. Replacing $Q$ with $R$ gives $P_{\left[ Q / R \right]} = R$ with intrinsic function $f_R$. Since $Q \equiv R$, it must be $f_Q = f_R$ and thus $P \equiv P_{\left[ Q / R \right]}$. This establishes the basic case.
  \item $P = \sequence{Q}{S}$ has the structure of sequential composition. Then:
    \begin{eqnarray*}
      \sem{P} & = & \sem{\sequence{Q}{S}} \\
              & \overset{\defref{def:sem_sequence}}{=} & \sem{S} \circ \sem{Q} \\
              & \overset{\sem{Q} = \sem{R}}{=} & \sem{S} \circ \sem{R} \\
              & \overset{\defref{def:sem_sequence}}{=} & \sem{sequence{R}{S}} \\
              & = & \sem{P_{\left[ Q / R \right]}}
    \end{eqnarray*}
    And thus $P \equiv P_{\left[ Q / R \right]}$.
  \item Analogously to (2), $P \equiv P_{\left[ Q / R \right]}$ can be shown for $P = \sequence{S}{Q}$.
  \item $P = \choice{Q}{S}{T}$ has the structure of choice composition. Then:
    \begin{eqnarray*}
      \sem{P} & = & \sem{\choice{Q}{S}{T}} \\
              & \overset{\defref{def:sem_choice}}{=} & x \mapsto \left\{ \begin{array}{ll}
                                                                           \sem{S} \left( x \right) & \text{if } Q \left( x \right) \\
                                                                           \sem{T} \left( x \right) & \text{if } \overline{Q \left( x \right)} \\
                                                                           \bot & \text{otherwise}
                                                                         \end{array}
                                                                 \right. \\
              & \overset{\sem{Q} = \sem{R}}{=} & x \mapsto \left\{ \begin{array}{ll}
                                                              \sem{S} \left( x \right) & \text{if } R \left( x \right) \\
                                                              \sem{T} \left( x \right) & \text{if } \overline{R \left( x \right)} \\
                                                              \bot & \text{otherwise}
                                                            \end{array}
                                                   \right. \\
              & \overset{\defref{def:sem_choice}}{=} & \sem{\choice{R}{S}{T}} \\
              & = & \sem{P_{\left[ Q / R \right]}}
    \end{eqnarray*}
    And thus $P \equiv P_{\left[ Q / R \right]}$.
    \item $P = \choice{S}{Q}{T}$ has the structure of choice composition. Then:
    \begin{eqnarray*}
      \sem{P} & = & \sem{\choice{S}{Q}{T}} \\
              & \overset{\defref{def:sem_choice}}{=} & x \mapsto \left\{ \begin{array}{ll}
                                                                           \sem{Q} \left( x \right) & \text{if } S \left( x \right) \\
                                                                           \sem{T} \left( x \right) & \text{if } \overline{S \left( x \right)} \\
                                                                           \bot & \text{otherwise}
                                                                         \end{array}
                                                                 \right. \\
              & \overset{\sem{Q} = \sem{R}}{=} & x \mapsto \left\{ \begin{array}{ll}
                                                              \sem{R} \left( x \right) & \text{if } S \left( x \right) \\
                                                              \sem{T} \left( x \right) & \text{if } \overline{S \left( x \right)} \\
                                                              \bot & \text{otherwise}
                                                            \end{array}
                                                   \right. \\
              & \overset{\defref{def:sem_choice}}{=} & \sem{\choice{S}{R}{T}} \\
              & = & \sem{P_{\left[ Q / R \right]}}
    \end{eqnarray*}
    And thus $P \equiv P_{\left[ Q / R \right]}$.
  \item Analogously to (5), $P \equiv P_{\left[ Q / R \right]}$ can be shown for $P = \choice{S}{T}{Q}$.
  \item $P = \parallel{Q}{S}{T}$ has the structure of parallel composition. Then:
    \begin{eqnarray*}
      \sem{P} & = & \sem{\parallel{Q}{S}{T}} \\
              & \overset{\defref{def:sem_parallel}}{=} & x \mapsto \sem{Q} \left( \sem{S}\left(x\right), \sem{T}\left(x\right) \right) \\
              & \overset{\sem{Q} = \sem{R}}{=} & x \mapsto \sem{R} \left( \sem{S}\left(x\right), \sem{T}\left(x\right) \right) \\
              & \overset{\defref{def:sem_parallel}}{=} & \sem{\parallel{R}{S}{T}} \\
              & = & \sem{P_{\left[ Q / R \right]}}
    \end{eqnarray*}
    And thus $P \equiv P_{\left[ Q / R \right]}$.
  \item $P = \parallel{S}{Q}{T}$ has the structure of parallel composition. Then:
    \begin{eqnarray*}
      \sem{P} & = & \sem{\parallel{S}{Q}{T}} \\
              & \overset{\defref{def:sem_parallel}}{=} & x \mapsto \sem{S} \left( \sem{Q}\left(x\right), \sem{T}\left(x\right) \right) \\
              & \overset{\sem{Q} = \sem{R}}{=} & x \mapsto \sem{S} \left( \sem{R}\left(x\right), \sem{T}\left(x\right) \right) \\
              & \overset{\defref{def:sem_parallel}}{=} & \sem{\parallel{S}{R}{T}} \\
              & = & \sem{P_{\left[ Q / R \right]}}
    \end{eqnarray*}
    And thus $P \equiv P_{\left[ Q / R \right]}$.
  \item Analogously to (8), $P \equiv P_{\left[ Q / R \right]}$ can be shown for $P = \parallel{S}{T}{Q}$.
  \item $P = \repetition{Q}{S}$ has the structure of repetition. Then $P \equiv P_{\left[ Q / R \right]}$ since repetition is expressed in terms of choice and sequential composition and process equivalence for these cases has been shown in (2), (3), (4), (5), (6).
  \item $P = \repetition{S}{Q}$ has the structure of repetition. Then $P \equiv P_{\left[ Q / R \right]}$ since repetition is expressed in terms of choice and sequential composition and process equivalence for these cases has been shown in (2), (3), (4), (5), (6).
\end{enumerate}
In all cases $P \equiv P_{\left[ Q / R \right]}$ holds. Thus, \thmref{thm:substitution_equivalence} is valid.

\hfill$\blacksquare$
\end{myproof}