\chapter{Proofs}
\label{chp:proofs}

\begin{theorem}[Associative property of sequence composition]
\label{thm:associativity_sequence}
Sequence composition has associative property, i.e. for any three processes $P, Q, R \in \mathcal{P}$ with suitable type signatures, the following equivalence holds:
\begin{equation*}
  \sequence{P}{\sequence{Q}{R}} \equiv \sequence{\sequence{P}{Q}}{R}.
\end{equation*}
\end{theorem}

\begin{myproof}[Associative property of sequence composition]
Let $P, Q, R \in \mathcal{P}$ be processes with type signatures $\typesignature{P}{a}{b}, \typesignature{Q}{b}{c}, \typesignature{R}{c}{d}$.

Then $\sem{\sequence{P}{\sequence{Q}{R}}}{x} = \sem{\sequence{\sequence{P}{Q}}{R}}{x}$.
\begin{eqnarray*}
  \sem{\sequence{P}{\sequence{Q}{R}}}{x} & \overset{\defref{def:sem_sequence}}{=} & \sem{\sequence{Q}{R}}{\sem{P}{x}} \\
                                         & \overset{\defref{def:sem_sequence}}{=} & \sem{R}{\sem{Q}{\sem{P}{x}}} \\
                                         & \overset{\defref{def:sem_sequence}}{=} & \sem{R}{\sem{\sequence{P}{Q}}{x}} \\
                                         & \overset{\defref{def:sem_sequence}}{=} & \sem{\sequence{\sequence{P}{Q}}{R}}{x}
\end{eqnarray*}

\hfill$\blacksquare$
\end{myproof}

\begin{theorem}[Distributivity of parallel composition over choice composition]
\label{thm:distributivity_parallel_choice}
Parallel composition distributes over choice composition, i.e. for any processes $B \in \mathcal{B}$ and $C, P, Q, R \in \mathcal{P}$ with suitable type signatures, the following equivalence holds:
\begin{equation*}
  \parallel{C}{P}{\choice{B}{Q}{R}} \equiv \choice{B}{\parallel{C}{P}{Q}}{\parallel{C}{P}{R}}.
\end{equation*}
\end{theorem}

\clearpage
\begin{myproof}[Distributivity of parallel composition over choice composition]
Let $B \in \mathcal{B}$ and $C, P, Q, R \in \mathcal{P}$ be processes with type signatures $\typesignature{B}{a}{T_{Boolean}}, \typesignature{C}{\left( b, c \right)}{d}, \typesignature{P}{a}{b}, \typesignature{Q}{a}{c}, \typesignature{R}{a}{c}$.

Then $\sem{\parallel{C}{P}{\choice{B}{Q}{R}}}{x} = \sem{\choice{B}{\parallel{C}{P}{Q}}{\parallel{C}{P}{R}}}{x}$.
\begin{eqnarray*}
  &                                        & \sem{\parallel{C}{P}{\choice{B}{Q}{R}}}{x} \\
  & \overset{\defref{def:sem_parallel}}{=} & \sem{C}{\left( \sem{P}{x}, \sem{\choice{B}{Q}{R}}{x} \right)} \\
  & \overset{\defref{def:sem_choice}}{=}   & \sem{C}{\left( \sem{P}{x}, \left\{\begin{array}{ll}
                                                                                 \sem{Q}{x} & \text{if } B \left( x \right) \\
                                                                                 \sem{R}{x} & \text{if } \overline{B \left( x \right)}
                                                                               \end{array}
                                                                        \right.\right)} \\
  & =                                      & \left\{\begin{array}{ll}
                                                      \sem{C}{\left( \sem{P}{x}, \sem{Q}{x}} \right) & \text{if } B \left( x \right) \\
                                                      \sem{C}{\left( \sem{P}{x}, \sem{R}{x}} \right) & \text{if } \overline{B \left( x \right)}
                                                    \end{array}
                                             \right. \\
  & \overset{\defref{def:sem_parallel}}{=} & \left\{\begin{array}{ll}
                                                      \sem{\parallel{C}{P}{Q}}{x} & \text{if } B \left( x \right) \\
                                                      \sem{\parallel{C}{P}{R}}{x} & \text{if } \overline{B \left( x \right)}
                                                    \end{array}
                                             \right. \\
  & \overset{\defref{def:sem_choice}}{=}   & \sem{\choice{B}{\parallel{C}{P}{Q}}{\parallel{C}{P}{R}}}{x}
\end{eqnarray*}

\hfill$\blacksquare$
\end{myproof}

\begin{theorem}[Distributivity of sequence composition over choice composition]
\label{thm:distributivity_sequence_choice}
Sequence composition distributes over choice composition, i.e. for any processes $B \in \mathcal{B}$ and $P, Q, R \in \mathcal{P}$ with suitable type signatures, the following equivalence holds:
\begin{equation*}
  \sequence{P}{\choice{B}{Q}{R}} \equiv \choice{B}{\sequence{P}{Q}}{\sequence{P}{R}}
\end{equation*}
\end{theorem}

\begin{myproof}[Distributivity of sequence composition over choice composition]
Let $B \in \mathcal{B}$ and $P, Q, R \in \mathcal{P}$ be processes with type signatures $\typesignature{B}{a}{T_{Boolean}}, \typesignature{P}{a}{b}, \typesignature{Q}{b}{c}, \typesignature{R}{b}{c}$. 

Then $\sem{\sequence{P}{\choice{B}{Q}{R}}}{x} = \sem{\choice{B}{\sequence{P}{Q}}{\sequence{P}{R}}}{x}$.
\begin{eqnarray*}
  \sem{\sequence{P}{\choice{B}{Q}{R}}}{x} & \overset{\defref{def:sem_sequence}}{=} & \sem{\choice{B}{Q}{R}}{\sem{P}{x}} \\
                                          & \overset{\defref{def:sem_choice}}{=}   & \left\{\begin{array}{ll}
                                                                                              \sem{Q}{\sem{P}{x}} & \text{if } B \left( x \right) \\
                                                                                              \sem{R}{\sem{P}{x}} & \text{if } \overline{B \left( x \right)}
                                                                                            \end{array}
                                                                                     \right. \\
                                          & \overset{\defref{def:sem_sequence}}{=} & \left\{\begin{array}{ll}
                                                                                              \sem{\sequence{P}{Q}}{x} & \text{if } B \left( x \right) \\
                                                                                              \sem{\sequence{P}{R}}{x} & \text{if } \overline{B \left( x \right)}
                                                                                            \end{array}
                                                                                     \right. \\
                                          & \overset{\defref{def:sem_choice}}{=}   & \sem{\choice{B}{\sequence{P}{Q}}{\sequence{P}{R}}}{x}
\end{eqnarray*}


\hfill$\blacksquare$
\end{myproof}

\begin{theorem}[Distributivity of sequence over parallel]
\label{thm:distributivity_sequence_parallel}
The sequence operator $\triangleright$ distributes over the parallel operator $|$, i.e. for any processes $P, Q, R \in \mathcal{P}$ with suitable type signatures, the following equivalence holds:
\begin{equation*}
  P \triangleright \left( Q \,|\, R \right) \equiv \left( P \triangleright Q \right) \,|\, \left( P \triangleright R \right). 
\end{equation*}
\end{theorem}

\begin{myproof}[Distributivity of sequence over parallel]
Let $P, Q, R \in \mathcal{P}$ be processes with the type signatures $\rho \left( P \right) = \left( T_a, T_x \right)$ and $\rho \left( Q \right) = \rho \left( R \right) = \left( T_x, T_r \right)$ for some $a, r, x$ and $T_a, T_r, T_x \in \mathcal{T}$. Let $\left( T_r, \star \right)$ be a semi-group with the commutative binary operation $\star \colon T_r \times T_r \to T_r$. Then $sem \left\langle P \triangleright \left( Q \,|\, R \right) \right\rangle = sem \left\langle \left( P \triangleright Q \right) | \left( P \triangleright R \right) \right\rangle$.
\begin{eqnarray*}
  sem \left\langle P \triangleright \left( Q \,|\, R \right) \right\rangle & \overset{\defref{def:sem_sequence}}{=} & \left\{ u \circ p \,|\, p \in sem \left\langle P \right\rangle, u \in sem \left\langle Q \,|\, R \right\rangle \right\} \\
  & \overset{\defref{def:sem_parallel}}{=} & \left\{ u \circ p \,|\, p \in sem \left\langle P \right\rangle, u \in \left\{q \star r \,|\, q \in sem \left\langle Q \right\rangle, r \in \left\langle R \right\rangle \right\} \right\} \\
  & \overset{flattening \ set}{=} & \left\{ \left( q \star r \right) \circ p \,|\, p \in sem \left\langle P \right\rangle, q \in sem \left\langle Q \right\rangle, r \in \left\langle R \right\rangle \right\} \\
  & \overset{???}{=} & \left\{ \left( q \circ p \right) \star \left( r \circ p \right) \,|\, p \in sem \left\langle P \right\rangle, q \in sem \left\langle Q \right\rangle, r \in \left\langle R \right\rangle \right\} \\
  & \overset{\defref{def:sem_sequence}}{=} & \left\{ v \star w \,|\, v \in sem \left\langle P \triangleright Q \right\rangle, w \in sem \left\langle P \triangleright R \right\rangle \right\} \\
  & \overset{\defref{def:sem_parallel}}{=} & sem \left\langle \left( P \triangleright Q \right) | \left( P \triangleright R \right) \right\rangle
\end{eqnarray*}
ToDo: This proof is broken! $\circ$ needs to distribute over $\star$ or can it work without?
\end{myproof}

\begin{theorem}[Commutativity of the choice operator]
\label{thm:commutativity_choice}
The choice operator $\vee$ has commutative properties, i.e. for any processes $P, Q \in \mathcal{P}$ with suitable type signatures, the following equivalence holds:
\begin{equation*}
  P \vee Q \equiv Q \vee P.
\end{equation*}
\end{theorem}

\begin{myproof}[Commutativity of the choice operator]
Let $P, Q \in \mathcal{P}$ be processes with the type signatures $\rho \left( P \right) = \rho \left( Q \right) = \left( T_a, T_r \right)$ for some $a, r$ and $T_a, T_r \in \mathcal{T}$. Then $sem \left\langle P \vee Q \right\rangle = sem \left\langle Q \vee P \right\rangle$.
\begin{eqnarray*}
  sem \left\langle P \vee Q \right\rangle & \overset{\defref{def:sem_choice}}{=} & sem \left\langle P \right\rangle \cup sem \left\langle Q \right\rangle \\
  & \overset{commutativity \ of \ \cup}{=} & sem \left\langle Q \right\rangle \cup sem \left\langle P \right\rangle \\
  & \overset{\defref{def:sem_choice}}{=} & sem \left\langle Q \vee P \right\rangle
\end{eqnarray*}
The commutative property of set union $\cup$ is transferred to the choice operator $\vee$ and hence, \thmref{thm:commutativity_choice} holds.

\hfill$\blacksquare$
\end{myproof}

\begin{theorem}[Commutativity of the parallel operator]
\label{thm:commutativity_parallel}
The parallel operator $|$ has commutative properties, i.e. for any processes $P, Q \in \mathcal{P}$ with suitable type signatures, the following equivalence holds:
\begin{equation*}
  P \,|\, Q \equiv Q \,|\, P.
\end{equation*}
\end{theorem}

\begin{myproof}[Commutativity of the parallel operator]
Let $P, Q \in \mathcal{P}$ be processes with the type signatures $\rho \left( P \right) = \rho \left( Q \right) = \left( T_a, T_r \right)$ for some $a, r$ and $T_a, T_r \in \mathcal{T}$. Let $\left( T_r, \star \right)$ be a semi-group with the commutative binary operation $\star \colon T_r \times T_r \to T_r$. Then $sem \left\langle P \,|\, Q \right\rangle = sem \left\langle Q \,|\, P \right\rangle$.
\begin{eqnarray*}
  sem \left\langle P \,|\, Q \right\rangle & \overset{\defref{def:sem_parallel}}{=} & \left\{ p \star q \,|\, p \in sem \left\langle P \right\rangle, q \in sem \left\langle Q \right\rangle \right\} \\
  & \overset{commutativity \ of \ \star}{=} & \left\{ q \star p \,|\, p \in sem \left\langle P \right\rangle, q \in sem \left\langle Q \right\rangle \right\} \\
  & \overset{\defref{def:sem_parallel}}{=} & sem \left\langle Q \,|\, P \right\rangle
\end{eqnarray*}
The commutative property of $\star$ is transferred to the parallel operator $|$ and hence, \thmref{thm:commutativity_parallel} holds.

\hfill$\blacksquare$
\end{myproof}

\begin{theorem}[Idempotence of the choice operator]
\label{thm:idempotence_choice}
When combining a process with itself using the choice operator $\vee$, it shows idempotence properties, i.e. for any process $P \in \mathcal{P}$, the following equivalence holds:
\begin{equation*}
  P \vee P \equiv P.
\end{equation*}
\end{theorem}

\begin{myproof}[Idempotence of the choice operator]
Let $P \in \mathcal{P}$ be a process, then $sem \left\langle P \vee P \right\rangle = sem \left\langle P \right\rangle$.
\begin{eqnarray*}
  sem \left\langle P \vee P \right\rangle & \overset{\defref{def:sem_choice}}{=} & sem \left\langle P \right\rangle \cup sem \left\langle P \right\rangle \\
  & \overset{union\ of\ equal\ sets}{=} & sem \left\langle P \right\rangle
\end{eqnarray*}
The idempotence property of $\cup$ for equal sets is transferred to the choice operator $\vee$ and hence, \thmref{thm:idempotence_choice} holds.

\hfill$\blacksquare$
\end{myproof}

\begin{theorem}[Idempotence of the sequence operator respecting the identity process]
\label{thm:idempotence_identity}
When the identity process $Id$ combined with itself sequentially, the resulting process is the identity process $Id$. The sequence operator is idempotent respecting the identity process:
\begin{equation*}
  Id \triangleright Id \equiv Id.
\end{equation*}
\end{theorem}

\begin{myproof}[Idempotence of the sequence operator respecting the identity process]
Let $Id \in \mathcal{P}$ be the identity process and use the sequence operator $\triangleright$ to compose it with itself. Then $sem \left\langle Id \triangleright Id \right\rangle = sem \left\langle Id \right\rangle$.
\begin{eqnarray*}
  sem \left\langle Id \triangleright Id \right\rangle & \overset{\defref{def:sem_id}}{=} & \left\{ \left( x \mapsto x \right) \circ \left( x \mapsto x \right) \right\} \\
  & \overset{function \ composition}{=} & \left\{ x \mapsto x \right\} \\
  & \overset{\defref{def:sem_id}}{=} & sem \left\langle Id \right\rangle
\end{eqnarray*}
The idempotence property of function composition $\circ$ respecting the identity function is transferred to sequential process composition $\triangleright$ and hence, \thmref{thm:idempotence_identity} holds.

\hfill$\blacksquare$
\end{myproof}

\begin{myproof}[Idempotence of the error process respecting the sequence operator]
\end{myproof}

\begin{myproof}[Idempotence of the \textsc{Kleene} star]
\end{myproof}

% ToDo: neutral elements