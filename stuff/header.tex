% $Id: header.tex 1229 2009-10-23 13s:58:42Z inf6254 $
%%%%%%%%%%%%%%%%%%%%%%%%%%%%%%%%%%%%%%%%%%%%%%%%%%%%%%%%%%%%%%%%%%%%%%%
\documentclass
  [ %twoside,            % beidseitiger Druck
   BCOR=10mm          % Bindekorrektur
  , openright          % Kapitel beginnen auf einer rechten Seite
  , listof=totoc       % Verzeichnisse im Inhaltsverzeichnis
  , bibliography=totoc % Literaturverzeichnis im Inhaltsverzeichnis
  , parskip=half       % Absätze durch einen vergrößerten Zeilenabstand getrennt
%  , draft              % Entwurfsversion
  ]{scrreprt}          % Dokumentenklasse: KOMA-Script Buch
  

%%%%%%%%%%%%%%%%%%%%%%%%%%%%%%%%%%%%%%%%%%%%%%%%%%%%%%%%%%%%%%%%%%%%%%%
% Packages
%%%%%%%%%%%%%%%%%%%%%%%%%%%%%%%%%%%%%%%%%%%%%%%%%%%%%%%%%%%%%%%%%%%%%%%
\usepackage{scrhack}
\usepackage[top=3cm,bottom=3cm]{geometry}

\usepackage{ifpdf}
\ifpdf
  \usepackage{ae}               % Fonts für pdfLaTeX, falls keine cm-super-Fonts installiert
  \usepackage{microtype}        % optischer Randausgleich, falls pdflatex verwandt
  \usepackage[pdftex]{graphicx} % Grafiken in pdfLaTeX
\else
  \usepackage[dvips]{graphicx}  % Grafiken und normales LaTeX
\fi

\usepackage[utf8]{inputenc}         % Input encoding (allow direct use of special characters like "ä")
\usepackage[english]{babel}
\usepackage[T1]{fontenc}
\usepackage[automark]{scrpage2} 	% Schickerer Satzspiegel mit KOMA-Script
\usepackage{setspace}           	% Allow the modification of the space between lines
\usepackage{booktabs}           	% Netteres Tabellenlayout
\usepackage{multicol}               % Mehrspaltige Bereiche
\usepackage{quotchap}               % Beautiful chapter decoration
\usepackage[printonlyused]{acronym} % list of acronyms and abbreviations
\usepackage{subfig}                 % allow sub figures
\usepackage{tabularx}              % Tabellen mit fester Breite
\usepackage{csquotes}
\usepackage{natbib}
\usepackage{array, amsmath, amssymb, amsthm}
\usepackage{lipsum}
\usepackage[toc,page]{appendix}
\usepackage[intoc]{nomencl}
\usepackage{makeidx}
\usepackage{lscape}

% for pictures
\usepackage{tikz}
\usetikzlibrary{positioning,shadows,arrows,shapes.geometric}
\newcommand{\figannotation}[1]{\tikz{\node[fill=black, inner sep = 2pt][circle] {\tiny\color{white}{#1}};}}
\newcommand{\smallfigannotation}[1]{\tikz{\node[fill=black, inner sep = 2pt, minimum size=10pt][circle] {\tiny\color{white}{#1}};}}

\theoremstyle{theorem}
\newtheorem{theorem}{Theorem}[chapter]
\newtheorem{definition}[theorem]{Definition}
\newtheorem{example}[theorem]{Example}
\newtheorem{corollary}[theorem]{Corollary}
\newtheorem{lemma}[theorem]{Lemma}
\newtheorem{guess}[theorem]{Guess}
\newtheorem{myproof}[theorem]{Proof}

% Layout
\pagestyle{scrheadings}
%\pagestyle{empty}
\clubpenalty = 10000
\widowpenalty = 10000
\displaywidowpenalty = 10000
\hbadness = 10000

\makeatletter
\renewcommand{\fps@figure}{htbp}
\makeatother

%% Document properties %%%%%%%%%%%%%%%%%%%%%%%%%%%%%%%%%%%%%%%%%%%%%%%%
\newcommand{\projname}{Hive}
\newcommand{\titel}{A process calculus\\ for\\ parallel and distributed programming\\ in Haskell}
\newcommand{\authorname}{Christopher Blöcker}
\newcommand{\thesisname}{Master Thesis}
\newcommand{\untertitel}{}
\newcommand{\Datum}{$2^\text{nd}$ of March 2015}
\newcommand{\chpref}[1]{chapter~\ref{#1}}
\newcommand{\eqnref}[1]{equation~\ref{#1}}
\newcommand{\expref}[1]{example~\ref{#1}}
\newcommand{\defref}[1]{definition~\ref{#1}}
\newcommand{\lstref}[1]{listing~\ref{#1}}
\newcommand{\thmref}[1]{theorem~\ref{#1}}
\newcommand{\crlref}[1]{corollary~\ref{#1}}
\newcommand{\appref}[1]{appendix~\ref{#1}}
\newcommand{\figref}[1]{figure~\ref{#1}}
\newcommand{\Figref}[1]{Figure~\ref{#1}}

\ifpdf
  \usepackage{hyperref}
  \definecolor{darkblue}{rgb}{0,0,.5}
  \hypersetup
  	{ colorlinks=true
  	, breaklinks=true
    , linkcolor=darkblue
    , menucolor=darkblue
    , urlcolor=darkblue
    , pdftitle={\projname -- \untertitel}
    , pdfsubject={\thesisname}
    , pdfauthor={\authorname}
    }
\else
\fi

%% shortcuts
\newcommand{\choiceop}{\vee}
\newcommand{\choice}[3]{\left( #1 \to #2 \choiceop #3 \right)}
\newcommand{\parallelop}{|}
\renewcommand{\parallel}[3]{\left( #1 \gets #2 \, \parallelop \, #3 \right)}
\newcommand{\sequenceop}{\triangleright}
\newcommand{\sequence}[2]{\left( #1 \sequenceop #2 \right)}
\newcommand{\repetitionop}{\propto}
\newcommand{\repetition}[2]{\left( #1 \repetitionop #2 \right)}

\newcommand{\typesignature}[3]{\rho \left( #1 \right) = \left( #2, #3 \right)}
\newcommand{\sem}[2]{sem \left\langle #1, #2 \right\rangle}

%% layout for nomenclature
\setlength{\nomlabelwidth}{.20\hsize}
\renewcommand{\nomlabel}[1]{#1 \dotfill}
\setlength{\nomitemsep}{-\parsep}

%% Listings %%%%%%%%%%%%%%%%%%%%%%%%%%%%%%%%%%%%%%%%%%%%%%%%%%%%%%%%%
\usepackage{listings}
\KOMAoptions{listof=totoc} % necessary because of scrhack
\renewcommand{\lstlistlistingname}{List of Listings}
\lstset
  { basicstyle=\small\ttfamily
  , breaklines=true
  , captionpos=b
  , showstringspaces=false
  , keywordstyle={}
  }

\lstnewenvironment{inlinehaskell}
{\spacing{1}\lstset{language=haskell,nolol,aboveskip=\bigskipamount}}
{\endspacing}

\lstnewenvironment{inlinexml}
{\spacing{1}\lstset{language=XML,nolol,aboveskip=\bigskipamount}}
{\endspacing}

\newcommand{\haskellinput}[2][]{
  \begin{spacing}{1}
  \lstinputlisting[language=Haskell,nolol,aboveskip=\bigskipamount,#1]{#2}
  \end{spacing}
}

\newcommand{\haskellcode}[2][]{\mylisting[#1,language=Haskell]{#2}}

\newcommand{\mylisting}[2][]{
\begin{spacing}{1}
\lstinputlisting[frame=lines,aboveskip=2\bigskipamount,#1]{#2}
\end{spacing}
}

\makeindex
\makenomenclature